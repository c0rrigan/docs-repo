\documentclass[letterpaper]{article}
\usepackage[T1]{fontenc}
\usepackage[utf8]{inputenc}
\usepackage{tocloft,siunitx,amssymb,amsmath,graphicx,subcaption}
\usepackage[top=3cm,left=3cm,right=3cm]{geometry}
\usepackage{float,pgf,pgfplots}
\usepackage[american]{circuitikz}
\graphicspath{{img/}}
\renewcommand\cftsecfont{\normalfont}
\renewcommand\cftsecpagefont{\normalfont}
\renewcommand{\cftsecleader}{\cftdotfill{\cftsecdotsep}}
\renewcommand\cftsecdotsep{\cftdot}
\renewcommand\cftsubsecdotsep{\cftdot}
\renewcommand\cftsubsubsecdotsep{\cftdot}
\title{Lab :}
\author{
    Sebastián Nava López\\
    \and
    Ericka Sabrina Pensamiento R.\\
    \and
    Salvador Palos Gil
}
%\captionsetup[subfigure]{justification=raggedright}
\begin{document}
\begin{titlepage}
    \centering
    {\Huge Instituto Politécnico Nacional}\\[3ex]
    {\huge Escuela Superior de Cómputo}\\[8ex]
    {\huge Fundamental Circuit Analysis}\\[12ex]
    {\Large Lab : }\\[20ex]
    {\Large Group: 1CV5 Team: 7 \\[8ex]
    Sebastian Nava López\\[4ex]
    Sabrina Erika Pensamiento Robledo\\[4ex]
    Salvador Palos Gil\\[18ex]
    }
    \large{Elaboration:April 17,2018 \hspace{8em} Due date:April 24,2018}
\end{titlepage}
\tableofcontents
\newpage
\section{Introduction}
\newpage
\section{Development}
\subsubsection{Verification of mesh analysis in DC}
Using a circuit as the one shown in fig: 
%\begin{figure}
%    \centering
%\begin{circuitikz}
%    \draw
%        (0,2) to [V](0,0)
%        (0,2) -- (0,4)
%        (0,4) to [R=$\SI{680}{\ohm}(R_1)$](3,4)
%        to [V](6,4) to
%        (6,2) to [R,*-](3,2)
%        to [R,*-*](0,2)
%        (6,2) to [R=$\SI{1}{\kilo\ohm}(R_5)$](6,0) -- (0,0)
%        (3,0) to [R=$\SI{470}{\ohm}(R_4)$,*-](3,2)
%        {
%            [anchor = south east](5.4,2.2) node {$\SI{560}{\ohm}(R_3)$}
%            [anchor = south east](2.4,2.2) node {$\SI{1}{\kilo\ohm}(R_2)$}
%            [anchor = south east](0,1.4) node {$\SI{9}{\volt}(V_{S1})$}
%            [anchor = south east](5.4,4.4) node {$\SI{5}{\volt}(V_{S2})$}
%        }
%        ;
%\end{circuitikz}
%\end{figure}
we measured current in each mesh ,and voltage values in each resistance.
\subsubsection{Calculations}
Using mesh analysis to calculate currents, we assign clockwise direction to every current, as
shown:
\begin{figure}
    \centering
\begin{circuitikz}
    \draw
        (0,2) to [V](0,0)
        (0,2) -- (0,4)
        (0,4) to [R=$R_1$](3,4)
        to [short,i_=$I_1$](3.5,4)
        to [V](6,4) to
        (6,2) to [R,*-](3,2)
        to [R,*-*](0,2)
        (6,2) to [R=$R_5$](6,0) -- (0,0)
        (3,0) to [R=$R_4$,*-](3,2)
        (3,0) to [short,i_=$I_2$](0,0)
        (6,0) to [short,i_=$I_3$](3,0)
        {
            [anchor = south east](5.4,2.2) node {$R_3$}
            [anchor = south east](2.4,2.2) node {$R_2$}
            [anchor = south east](0,1.4) node {$V_{S1}$}
            [anchor = south east](5.4,4.4) node {$V_{S2}$}
        }
        ;
\end{circuitikz}
\end{figure}
\begin{figure}
    \centering
\begin{tabular}{|c|c|c|c|}\hline
    Measurements & Theoretical Value(\si{\milli\ampere}) & Measured Value(\si{\milli\ampere}) &
    Simulated Value (\si{\milli\ampere})\\\hline
    $I_{1}$ & & & \\\hline
    $I_{2}$ & & & \\\hline
    $V_{R1}$ & & & \\\hline 
    $V_{R2}$ & & & \\\hline 
    $V_{R3}$ & & & \\\hline 
    $V_{R4}$ & & & \\\hline 
    $V_{R5}$ & & & \\\hline 
\end{tabular}
\end{figure}
\section{Questions}
\textit{\textbf{Explain the operation of the oscilloscope}}\\
%ans
\textit{\textbf{What's the function of the signal generator?}}\\
%ans
\textit{\textbf{What are the Lissajous graphs for?}}\\
%ans
\textit{\textbf{What are the operating modes Y (t) and XY used for?}}\\
%ans
\textit{\textbf{Which means by coupling in D.C.?}}\\
%ans
\textit{\textbf{What is an offset signal?}}\\
%ans
\textit{\textbf{Which means that a signal is out of phase?}}\\
%ans
\section{Conclusions}
{\large Sabrina:}\\
%
\\[2ex]
{\large Salvador:}\\
%
\\[2ex]
{\large Sebastián:}\\
%texto
\end{document}
