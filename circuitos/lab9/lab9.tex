\documentclass[letterpaper]{article}
\usepackage[T1]{fontenc}
\usepackage[utf8]{inputenc}
\usepackage{tocloft,siunitx,amssymb,amsmath,graphicx,subcaption}
\usepackage[top=3cm,left=3cm,right=3cm]{geometry}
\usepackage{float,pgf,pgfplots}
\usepackage[american]{circuitikz}
\graphicspath{{img/}}
\renewcommand\cftsecfont{\normalfont}
\renewcommand\cftsecpagefont{\normalfont}
\renewcommand{\cftsecleader}{\cftdotfill{\cftsecdotsep}}
\renewcommand\cftsecdotsep{\cftdot}
\renewcommand\cftsubsecdotsep{\cftdot}
\renewcommand\cftsubsubsecdotsep{\cftdot}
\title{Lab 9: Thévenin's Theorem}
\author{
    Sebastián Nava López\\
    \and
    Ericka Sabrina Pensamiento R.\\
    \and
    Salvador Palos Gil
}
%\captionsetup[subfigure]{justification=raggedright}
\begin{document}
\begin{titlepage}
    \centering
    {\Huge Instituto Politécnico Nacional}\\[3ex]
    {\huge Escuela Superior de Cómputo}\\[8ex]
    {\huge Fundamental Circuit Analysis}\\[12ex]
    {\Large Lab 9: Thévenin's Theorem}\\[20ex]
    {\Large Group: 1CV5 Team: 7 \\[8ex]
    Sebastian Nava López\\[4ex]
    Sabrina Erika Pensamiento Robledo\\[4ex]
    Salvador Palos Gil\\[18ex]
    }
    \large{Elaboration:?? \hspace{8em} Due date:??}
\end{titlepage}
\tableofcontents
\newpage
\section{Introduction}
\newpage
\section{Development}
Using the circuit shown in Fig.\ref{fig:diag1}:
\begin{figure}[H]
    \centering
    \begin{circuitikz}[scale=0.75,transform shape]
        \draw (0,0) to [V = $\SI{5}{\volt}$,invert] (0,3)
        to [R = $\SI{1}{\kilo\ohm}$] (2,3)
        to [R = $\SI{470}{\ohm}$] (4,3)
        to [V = $\SI{20}{\volt}$,invert] (6,3)
        to [R = $\SI{1}{\kilo\ohm}$] (8,3)
        to [V = $\SI{10}{\volt}$,invert] (10,3) 
        to [short,-*](11,3)
        (2,3) to [R = $\SI{1}{\kilo\ohm}$] (2,0)
        (8,3) to [R = $\SI{2.2}{\kilo\ohm}$] (8,0)
        (0,0) to [short,-*] (11,0);
        \draw {
            [anchor = south west](11,0) node {\textbf{A}}
            [anchor = south west](11,3) node {\textbf{B}}
        };
    \end{circuitikz}
    \caption{Circuit diagram}
    \label{fig:diag1}
\end{figure}
first we need to find Thévenin-equivalent resistance($R_{TH}$), in order to find it we need to
replace all the voltage sources with a short circuit, the resulting circuit is the one shown in
Fig.\ref{fig:diag2}:
\begin{figure}[H]
    \centering
    \begin{circuitikz}[scale=0.75,transform shape]
        \draw (0,0) -- (0,3)
        to [R = $\SI{1}{\kilo\ohm}$] (2,3)
        to [R = $\SI{470}{\ohm}$] (4,3)
        to [R = $\SI{1}{\kilo\ohm}$] (6,3)
        to [short,-*](9,3)
        (2,3) to [R = $\SI{1}{\kilo\ohm}$] (2,0)
        (6,3) to [R = $\SI{2.2}{\kilo\ohm}$] (6,0)
        (0,0) to [short,-*] (9,0);
        \draw {
            [anchor = south west](9,0) node {\textbf{A}}
            [anchor = south west](9,3) node {\textbf{B}}
        };
    \end{circuitikz}
    \caption{Resulting circuit after shorting the power sources}
    \label{fig:diag2}
\end{figure}
First we combine the \SI{1}{\kilo\ohm} and \SI{470}{\ohm} resistors located at the center of the
circuit, as they are connected in series, the equivalent resistor is equal to the sum of both
resistances.
\begin{figure}[H]
    \centering
    \begin{circuitikz}[scale=0.75,transform shape]
        \draw (0,0) -- (0,3)
        to [R = $\SI{1}{\kilo\ohm}$] (2,3)
        to [R = $\SI{1.47}{\kilo\ohm}$] (4,3)
        to [short,-*](7,3)
        (2,3) to [R = $\SI{1}{\kilo\ohm}$] (2,0)
        (4,3) to [R = $\SI{2.2}{\kilo\ohm}$] (4,0)
        (0,0) to [short,-*] (7,0);
        \draw {
            [anchor = south west](7,0) node {\textbf{A}}
            [anchor = south west](7,3) node {\textbf{B}}
        };
    \end{circuitikz}
\end{figure}
Then, we combine the \SI{1}{\kilo\ohm} and \SI{1}{\kilo\ohm} resistors that are wired in parellel,
the equivalent resistor is equal to
$\frac{\SI{1}{\kilo\ohm}\cdot\SI{1}{\kilo\ohm}}{2(\SI{1}{\kilo\ohm})} = \SI{500}{\ohm}$
\begin{figure}[H]
    \centering
    \begin{circuitikz}[scale=0.75,transform shape]
        \draw (0,0) -- (0,3)
        to [R = $\SI{500}{\ohm}$] (2,3)
        to [R = $\SI{1.47}{\kilo\ohm}$] (4,3)
        to [short,-*](7,3)
        (4,3) to [R = $\SI{2.2}{\kilo\ohm}$] (4,0)
        (0,0) to [short,-*] (7,0);
        \draw {
            [anchor = south west](7,0) node {\textbf{A}}
            [anchor = south west](7,3) node {\textbf{B}}
        };
    \end{circuitikz}
\end{figure}
Combining the \SI{500}{\ohm} and \SI{1.47}{\kilo\ohm} resistors, the equivalent resistor is equal
to $\SI{500}{\ohm}+\SI{1.47}{\kilo\ohm} = \SI{1.97}{\kilo\ohm}$
\begin{figure}[H]
    \centering
    \begin{circuitikz}[scale=0.75,transform shape]
        \draw (0,0) -- (0,3)
        to [R = $\SI{1.97}{\kilo\ohm}$] (2,3)
        to [short,-*](5,3)
        (2,3) to [R = $\SI{2.2}{\kilo\ohm}$] (2,0)
        (0,0) to [short,-*] (5,0);
        \draw {
            [anchor = south west](5,0) node {\textbf{A}}
            [anchor = south west](5,3) node {\textbf{B}}
        };
    \end{circuitikz}
\end{figure}
Finally, $R_{TH}$ is given by the equivalent resistance of the \SI{1.97}{\kilo\ohm} and
\SI{2.2}{\kilo\ohm} resistors connected in parallel:
\[\therefore R_{TH} =
\frac{\SI{1.97}{\kilo\ohm}\cdot\SI{2.2}{\kilo\ohm}}{\SI{1.97}{\kilo\ohm}+\SI{2.2}{\kilo\ohm}}\ =
\SI{1039.328}{\ohm}\]
To find the Thévenin-equivalent voltage we need to find the voltage passing in nodes $A$ and $B$,
we choose to simplify the circuit using the source transformation method.
First we transform the \SI{5}{\volt} source to a current source:
\[I_S = \frac{\SI{5}{\volt}}{\SI{1}{\kilo\ohm}}\ = \SI{5}{\milli\ampere}\]
\begin{figure}[H]
    \centering
    \begin{circuitikz}[scale=0.75,transform shape]
        \draw (0,0) to [I,label = $\SI{5}{\milli\ampere}$] (0,3) -- (4,3)
        (4,3) to [R = $\SI{470}{\ohm}$] (6,3)
        to [V = $\SI{20}{\volt}$,invert] (8,3)
        to [R = $\SI{1}{\kilo\ohm}$] (10,3)
        to [V = $\SI{10}{\volt}$,invert] (12,3) 
        to [short,-*](13,3)
        (2,3) to [R = $\SI{1}{\kilo\ohm}$] (2,0)
        (4,3) to [R = $\SI{1}{\kilo\ohm}$] (4,0)
        (10,3) to [R = $\SI{2.2}{\kilo\ohm}$] (10,0)
        (0,0) to [short,-*] (13,0);
        \draw {
            [anchor = south west](13,0) node {\textbf{A}}
            [anchor = south west](13,3) node {\textbf{B}}
        };
    \end{circuitikz}
\end{figure}
Merging the \SI{1}{\kilo\ohm} resistor and the \SI{1}{\kilo\ohm} resistor connected in parellel yields a
equivalent resistor equal to:
\[R_{eq} = \frac{(\SI{1}{\kilo\ohm})(\SI{1}{\kilo\ohm})}{2(\SI{1}{\kilo\ohm})} = \SI{500}{\ohm}\]
\begin{figure}[H]
    \centering
    \begin{circuitikz}[scale=0.75,transform shape]
        \draw (0,0) to [I,label = $\SI{5}{\milli\ampere}$] (0,3) -- (2,3)
        (2,3) to [R = $\SI{470}{\ohm}$] (4,3)
        to [V = $\SI{20}{\volt}$,invert] (6,3)
        to [R = $\SI{1}{\kilo\ohm}$] (8,3)
        to [V = $\SI{10}{\volt}$,invert] (10,3) 
        to [short,-*](11,3)
        (2,3) to [R = $\SI{500}{\ohm}$] (2,0)
        (8,3) to [R = $\SI{2.2}{\kilo\ohm}$] (8,0)
        (0,0) to [short,-*] (11,0);
        \draw {
            [anchor = south west](11,0) node {\textbf{A}}
            [anchor = south west](11,3) node {\textbf{B}}
        };
    \end{circuitikz}
\end{figure}
Next we transform the \SI{5}{\milli\ampere} current source to a voltage source:
\[V_S = \SI{5}{\milli\ampere}\cdot\SI{500}{\ohm} = \SI{2.5}{\volt}\]
\begin{figure}[H]
    \centering
    \begin{circuitikz}[scale=0.75,transform shape]
        \draw (0,0) to [V,label = $\SI{2.5}{\volt}$,invert] (0,3)
        to [R = $\SI{500}{\ohm}$] (2,3)
        to [R = $\SI{470}{\ohm}$] (4,3)
        to [V = $\SI{20}{\volt}$,invert] (6,3)
        to [R = $\SI{1}{\kilo\ohm}$] (8,3)
        to [V = $\SI{10}{\volt}$,invert] (10,3) 
        to [short,-*](11,3)
        (8,3) to [R = $\SI{2.2}{\kilo\ohm}$] (8,0)
        (0,0) to [short,-*] (11,0);
        \draw {
            [anchor = south west](11,0) node {\textbf{A}}
            [anchor = south west](11,3) node {\textbf{B}}
        };
    \end{circuitikz}
\end{figure}
Combining the \SI{500}{\ohm} and \SI{470}{\ohm} resistors connected in series:   
\[R_{EQ} = \SI{500}{\ohm}+\SI{470}{\ohm} = \SI{970}{\ohm}\]
\begin{figure}[H]
    \centering
    \begin{circuitikz}[scale=0.75,transform shape]
        \draw (0,0) to [V,label = $\SI{2.5}{\volt}$,invert] (0,3)
        to [R = $\SI{970}{\ohm}$] (2,3)
        to [V = $\SI{20}{\volt}$,invert] (4,3)
        to [R = $\SI{1}{\kilo\ohm}$] (6,3)
        to [V = $\SI{10}{\volt}$,invert] (8,3) 
        to [short,-*](9,3)
        (6,3) to [R = $\SI{2.2}{\kilo\ohm}$] (6,0)
        (0,0) to [short,-*] (9,0);
        \draw {
            [anchor = south west](9,0) node {\textbf{A}}
            [anchor = south west](9,3) node {\textbf{B}}
        };
    \end{circuitikz}
\end{figure}
Changing the order of the components and combining resistors and voltage sources we have:
\begin{gather*}
    V_S = \SI{22.5}{\volt}\\
    R_{EQ} = \SI{1}{\kilo\ohm}+\SI{970}{\ohm} = \SI{1.97}{\kilo\ohm}
\end{gather*}
\begin{figure}[H]
    \centering
    \begin{circuitikz}[scale=0.75,transform shape]
        \draw (0,0) to [V,label = $\SI{22.5}{\volt}$,invert] (0,3)
        to [R = $\SI{1.97}{\kilo\ohm}$] (3,3)
        to [V = $\SI{10}{\volt}$,invert] (6,3) 
        to [short,-*](6,3)
        (3,3) to [R = $\SI{2.2}{\kilo\ohm}$] (3,0)
        (0,0) to [short,-*] (6,0);
        \draw {
            [anchor = south west](6,0) node {\textbf{A}}
            [anchor = south west](6,3) node {\textbf{B}}
        };
    \end{circuitikz}
\end{figure}
Transforming the voltage source into a current source:
\[I_S = \frac{\SI{22.5}{\volt}}{\SI{1.97}{\kilo\ohm}} = \SI{11.421}{\milli\ampere}\]
\begin{figure}[H]
    \centering
    \begin{circuitikz}[scale=0.75,transform shape]
        \draw (0,0) to [I,label = $\SI{11.421}{\milli\ampere}$] (0,3) -- (4,3)
        (4,3) to [V = $\SI{10}{\volt}$,invert] (7,3) 
        to [short,-*](7,3)
        (2,3) to [R = $\SI{1.97}{\kilo\ohm}$] (2,0)
        (4,3) to [R = $\SI{2.2}{\kilo\ohm}$] (4,0)
        (0,0) to [short,-*] (7,0);
        \draw {
            [anchor = south west](7,0) node {\textbf{A}}
            [anchor = south west](7,3) node {\textbf{B}}
        };
    \end{circuitikz}
\end{figure}
Combining the parallel resistors we have that:
\[R_{EQ} =
\frac{\SI{1.97}{\kilo\ohm}\cdot\SI{2.2}{\kilo\ohm}}{\SI{1.97}{\kilo\ohm}+\SI{2.2}{\kilo\ohm}} =
\SI{1039.328}{\ohm}\]
\begin{figure}[H]
    \centering
    \begin{circuitikz}[scale=0.75,transform shape]
        \draw (0,0) to [I,label = $\SI{11.421}{\milli\ampere}$] (0,3) -- (2,3)
        (2,3) to [V = $\SI{10}{\volt}$,invert] (5,3) 
        to [short,-*](5,3)
        (2,3) to [R = $\SI{1039}{\ohm}$] (2,0)
        (0,0) to [short,-*] (5,0);
        \draw {
            [anchor = south west](5,0) node {\textbf{A}}
            [anchor = south west](5,3) node {\textbf{B}}
        };
    \end{circuitikz}
\end{figure}
Transforming the current source to a voltage source:
\[V_S = (\SI{11.421}{\milli\ampere})(\SI{1039.328}{\ohm}) = \SI{11.870}{\volt}\]
\begin{figure}[H]
    \centering
    \begin{circuitikz}[scale=0.75,transform shape]
        \draw (0,0) to [V,label = $\SI{11.87}{\volt}$,invert] (0,3) 
        to [R = $\SI{1039}{\ohm}$](2,3)
        to [V = $\SI{10}{\volt}$,invert] (5,3) 
        to [short,-*](5,3)
        (0,0) to [short,-*] (5,0);
        \draw {
            [anchor = south west](5,0) node {\textbf{A}}
            [anchor = south west](5,3) node {\textbf{B}}
        };
    \end{circuitikz}
\end{figure}
Then we change the order of series elements and merging both voltage sources:
\[V_S = \SI{11.870}{\volt}+\SI{10}{\volt} = \SI{21.870}{\volt}\]
\begin{figure}[H]
    \centering
    \begin{circuitikz}[scale=0.75,transform shape]
        \draw (0,0) to [V,label = $\SI{21.87}{\volt}$,invert] (0,3) 
        to [R = $\SI{1039}{\ohm}$,-*](4,3)
        (0,0) to [short,-*] (4,0);
        \draw {
            [anchor = south west](4,0) node {\textbf{A}}
            [anchor = south west](4,3) node {\textbf{B}}
        };
    \end{circuitikz}
\end{figure}
Finally, the Thévenin-equivalent voltage is equal to:
\[\therefore V_{TH} = \SI{21.87}{\volt}\]
The current going along nodes $A$ and $B$ is given by:
\[I_{AB} = \frac{V_{TH}}{R_{TH}} = \frac{\SI{21.870}{\volt}}{\SI{1039.328}{\ohm}} =
\SI{21.649}{\milli\ampere}\]
When the load resistor($R_L$) is connected, the voltage in the resistor is given by:
\[V_L = R_L\Bigg(\frac{V_{TH}}{R_{TH}+R_L}\Bigg) =
\SI{3.3}{\kilo\ohm}\Bigg(\frac{\SI{21.870}{\volt}}{\SI{1039.328}{\ohm}+\SI{3.3}{\kilo\ohm}}\Bigg)
= \SI{16.632}{\volt}\]
The current in the resistor is:
\[I_L = \frac{V_L}{R_L} = \frac{\SI{16.632}{\volt}}{\SI{3.3}{\kilo\ohm}} =
\SI{5.040}{\milli\ampere}\]
Power in the resistor is given by:
\[P_L = I_L\cdot\ V_L = (\SI{5.040}{\milli\ampere})(\SI{16.632}{\volt}) = \SI{0.083}{\watt}\]
\section{Questions}
\textit{\textbf{Explain the operation of the oscilloscope}}\\
%ans
\textit{\textbf{What's the function of the signal generator?}}\\
%ans
\textit{\textbf{What are the Lissajous graphs for?}}\\
%ans
\textit{\textbf{What are the operating modes Y (t) and XY used for?}}\\
%ans
\textit{\textbf{Which means by coupling in D.C.?}}\\
%ans
\textit{\textbf{What is an offset signal?}}\\
%ans
\textit{\textbf{Which means that a signal is out of phase?}}\\
%ans
\section{Conclusions}
{\large\textbf{Sabrina:}}\\
%
\\[2ex]
{\large\textbf{Salvador:}}\\
%
\\[2ex]
{\large\textbf{Sebastián:}}\\
%texto
\end{document}
