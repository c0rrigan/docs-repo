\documentclass[a4paper]{article}
\usepackage[T1]{fontenc}
\usepackage[utf8]{inputenc}
\usepackage{graphicx,tocloft,siunitx,amssymb,amsmath,float}
%\usepackage{subcaption,pgf,pgfplots}
\usepackage[top=3cm,left=3cm,right=3cm]{geometry}
%\graphicspath{{img/}}
\renewcommand\cftsecfont{\normalfont}
\renewcommand\cftsecpagefont{\normalfont}
\renewcommand{\cftsecleader}{\cftdotfill{\cftsecdotsep}}
\renewcommand\cftsecdotsep{\cftdot}
\renewcommand\cftsubsecdotsep{\cftdot}
\renewcommand\cftsubsubsecdotsep{\cftdot}
\title{Lab 3:Kirchhoff's Laws}
\author{
    Sebastián Nava López\\
    \and
    Ericka Sabrina Pensamiento R.\\
    \and
    Salvador Palos Gil
}
%\captionsetup[subfigure]{justification=raggedright}
\begin{document}
\begin{titlepage}
    \centering
    {\Huge Instituto Politécnico Nacional}\\[3ex]
    {\huge Escuela Superior de Cómputo}\\[8ex]
    {\huge Fundamental Circuit Analysis}\\[12ex]
    {\Large Lab 3:Kirchhoff's Laws}\\[20ex]
    {\Large Group: 1CV7 Team: 7 \\[8ex]
    Sebastian Nava López\\[4ex]
    Sabrina Erika Pensamiento Robledo\\[4ex]
    Salvador Palos Gil\\[18ex]
    }
    \large{Elaboration: March 13,2018\hspace{8em} Due date: March 20,2018}
\end{titlepage}
\tableofcontents
\newpage
\section{Introduction}
\newpage
\section{Development}
%----------Circuit1
\subsection{Verification of Kirchhoff's Voltage Law}
\subsubsection{Calculations}
In the first part, we wire the components in series as showed in the diagram:
%----------Circut 1 Diagram
Then, applying Kirchhoff's Voltage Law, we have that:
\begin{gather*}
    -V_{si}+VR_1+V_{s2}+VR_2+VR_3=0\\
    -9+470I+5+330I+56I=0\\
    1360I=9-5\\
    I=\frac{4}{1360}\si{\ampere}\qquad I=\SI{2.941}{\milli\ampere}
\end{gather*}
Now that we have the current value we can calculate the voltage value for each segment in the
circuit.\\
For $V_{0A}$:
\begin{align*}
    V_{0A}&=-V_{S1}\\
    V_{0A}&=-9V
\end{align*}
For $V_{AB}$:
\begin{align*}
    V_{AB}&=VR_1\\
    V_{AB}&=IR_1\\
    V_{AB}&=(\SI{2.941}{\milli\ampere})(\SI{470}{\ohm})\\
    V_{AB}&=\SI{1.382}{\volt}
\end{align*}
For $V_{BC}$:
\begin{align*}
    V_{BC}&=V_{S2}\\
    V_{BC}&=5V
\end{align*}
For $V_{CD}$:
\begin{align*}
    V_{CD}&=VR_2\\
    V_{CD}&=IR_2\\
    V_{CD}&=(\SI{2.941}{\milli\ampere})(\SI{330}{\ohm})\\
    V_{CD}&=\SI{0.970}{\volt}
\end{align*}
For $V_{D0}$:
\begin{align*}
    V_{D0}&=VR_3\\
    V_{D0}&=IR_1\\
    V_{D0}&=(\SI{2.941}{\milli\ampere})(\SI{560}{\ohm})\\
    V_{D0}&=\SI{1.647}{\volt}
\end{align*}
Finally, the sum of all voltages is:
\begin{align*}
    \sum\nolimits_{V}&=V_{0A}+V_{AB}+V_{BC}+V_{CD}+V_{D0}\\
    \sum\nolimits_{V}&=-\SI{9}{\milli\watt}+\SI{1.382}{\milli\watt}+\SI{5}{\milli\watt}+\SI{0.970}{\milli\watt}+\SI{1.647}{\milli\watt}\\
    \sum\nolimits_{V}&=\SI{-2.4e-4}{\milli\watt}\qquad
    \therefore\sum\nolimits_{V}\approx 0\si{\volt}
\end{align*}
Then, Power(P) is given by:
\[P=IV\]
Power $P_{0A}$ is equal to:
\begin{align*}
    P_{0A}&=IV_{0A}\\
    P_{0A}&=(\SI{2.941}{\milli\ampere})(\SI{-9}{\volt})\\
    P_{0A}&=\SI{-26.469}{\milli\watt}\\
\end{align*}
For $P_{AB}$:
\begin{align*}
    P_{AB}&=IV_{AB}\\
    P_{AB}&=(\SI{2.941}{\milli\ampere})(\SI{1.382}{\volt})\\
    P_{AB}&=\SI{4.064}{\milli\watt}\\
\end{align*}
For $P_{BC}$:
\begin{align*}
    P_{BC}&=IV_{BC}\\
    P_{BC}&=(\SI{2.941}{\milli\ampere})(\SI{5}{\volt})\\
    P_{BC}&=\SI{14.705}{\milli\watt}\\
\end{align*}
For $P_{CD}$:
\begin{align*}
    P_{CD}&=IV_{CD}\\
    P_{CD}&=(\SI{2.941}{\milli\ampere})(\SI{0.970}{\volt})\\
    P_{CD}&=\SI{2.853}{\milli\watt}\\
\end{align*}
For $P_{D0}$:
\begin{align*}
    P_{D0}&=IV_{D0}\\
    P_{D0}&=(\SI{2.941}{\milli\ampere})(\SI{1.647}{\volt})\\
    P_{D0}&=\SI{4.844}{\milli\watt}\\
\end{align*}
The sum of all the powers is:
\begin{align*}
    \sum\nolimits_{P}&=P_{0A}+P_{AB}+P_{BC}+P_{CD}+P_{D0}\\
    \sum\nolimits_{P}&=-\SI{-26.469}{\volt}+\SI{1.382}{\volt}+\SI{5}{\volt}+\SI{0.970}{\volt}+\SI{1.647}{\volt}\\
    \sum\nolimits_{P}&=\SI{-2.941e-6}{\volt}\qquad
    \therefore\sum\nolimits_{P}\approx \SI{0}{\watt}
\end{align*}
\subsubsection{Measurements}
\begin{table}[H]
    \centering
\begin{tabular}{|c|c|c|c|c|c|}\hline
    Measurements & Theoretical & Measured & Theoretical & Measured &
    Absorb(A)/\\
     &  Value (\si{\volt}) &  Value (\si{\volt}) & Power (\si{\milli\watt}) & Power (\si{\milli\watt}) & Supply(S)\\\hline
     Voltage $V_{0A}$ & 0 & 0 & 0 & 0 & 0\\\hline
     Voltage $V_{AB}$ & 0 & 0 & 0 & 0 & 0\\\hline
     Voltage $V_{BC}$ & 0 & 0 & 0 & 0 & 0\\\hline
     Voltage $V_{CD}$ & 0 & 0 & 0 & 0 & 0\\\hline
     Voltage $V_{D0}$ & 0 & 0 & 0 & 0 & 0\\\hline
\end{tabular}
\caption{Theoretical and experimental voltage values}\label{table:1}
\end{table}
\subsubsection{sim1}
%----------Circuit2
\subsection{Verification of Kirchhoff's Current Law}
In the next part we built the circuit according to this diagram:
%--------------------Diagram
We can see that,in contrast with the previous circuit, now we have two meshes meaning that we
have three different currents.In order to calculate the value of each current we must propose a
direction of current for every mesh and assign polarity depending on the proposed direction.Then
,using Kirchhoff's Current Law in the left mesh:
\begin{gather*}
    -V_{S1}+VR_1+VR_2=0\\
    -9+470I_1+330(I_1-I_2)=0
\end{gather*}\\[-7ex]
\begin{equation}-800I_1-330I_2=9\label{eq:1}\end{equation}
for the right mesh:
\begin{gather*}
    VR_2+VR_3+V_{S2}=0\\
    330(I_2-I_1)+560I_2=0
\end{gather*}\\[-7ex]
\begin{equation}-330I_1+890I_2=-5\label{eq:2}\end{equation}
With \eqref{eq:1} and  \eqref{eq:2} we have the next linear system:
\[
    \begin{cases}
        -800I_1-330I_2=9\\
        -330I_1+890I_2=-5
    \end{cases}
\]
Expressing the system in matrices:
\[
    \begin{bmatrix}
        -800 & -330\\
        -330 & 890
    \end{bmatrix}
    \begin{bmatrix}
        I_1\\
        I_2
    \end{bmatrix}
    =
    \begin{bmatrix}
        9\\
        -5
    \end{bmatrix}
\]
Using Crammer's rule to solve the linear system:
\begin{gather*}
    \Delta=
    \begin{bmatrix}
        -800 & -330\\
        -330 & 890
    \end{bmatrix}
    =-800\times 890-(-330)\times(-330)\\
\Delta=603100\\
    \begin{bmatrix}
        -800 & -9\\
        -330 & 5
    \end{bmatrix}
    =-800\times 5-(-9)\times(-330)\\
    \Delta_1=6360\\
\Delta_2=
    \begin{bmatrix}
        -9 & -330\\
        5 & 890
    \end{bmatrix}
    =-9\times890-(-330)\times(5)\\
    \Delta_2=-1030
\end{gather*}
Finally:
\begin{gather*}
    I_1=\frac{\Delta_1}{\Delta}=\frac{6360}{603100}\si{\ampere}\qquad I_1=\SI{10.545}{\milli\ampere}\\[2ex]
    I_2=\frac{\Delta_2}{\Delta}=\frac{-1030}{603100}\si{\ampere}\qquad I_2=\SI{-1.708}{\milli\ampere}\\
\end{gather*}
Using Kirchhoff's Current Law in node A:
%redibujar circuito
\begin{gather*}
I_1-I_2=I_3
\end{gather*}
The voltages for each segment are, for $V_{A0}$:
\begin{align*}
    V_{A0}&=V_{S1}\\
    V_{A0}&=\SI{9}{\volt}
\end{align*}
For $V_{AB}$:
\begin{align*}
    V_{AB}&=I_1R_1\\
    V_{AB}&=(\SI{10.545}{\milli\ampere})(\SI{470}{\ohm})\\
    V_{AB}&=\SI{4.956}{\volt}
\end{align*}
For $V_{B0}$:
\begin{align*}
    V_{B0}&=I_1R_1\\
    V_{B0}&=(\SI{10.545}{\milli\ampere})(\SI{330}{\ohm})\\
    V_{B0}&=\SI{4.043}{\volt}
\end{align*}
For $V_{BC}$:
\begin{align*}
    V_{BC}&=I_1R_1\\
    V_{BC}&=(\SI{10.545}{\milli\ampere})(\SI{470}{\ohm})\\
    V_{BC}&=\SI{-956.48}{\milli\volt}
\end{align*}
For $V_{C0}$:
\begin{align*}
    V_{C0}&=V_{S2}\\
    V_{C0}&=\SI{5}{\volt}
\end{align*}
Next, we calculate the power value for each segment of the circuit, in $P_{0A}$ we have:
\begin{align*}
    P_{0A}&=V_{0A}I_1\\
    P_{0A}&=(-\SI{9}{\volt})(\SI{10.545}{\milli\ampere})\\
    P_{0A}&=\SI{-94.905}{\milli\watt}
\end{align*}
In $P_{AB}$:
\begin{align*}
    P_{AB}&=V_{AB}I_1\\
    P_{AB}&=(\SI{4.956}{\volt})(\SI{10.545}{\milli\ampere})\\
    P_{AB}&=\SI{52.261}{\milli\watt}
\end{align*}
In $P_{B0}$:
\begin{align*}
    P_{B0}&=V_{B0}I_3\\
    P_{B0}&=(\SI{4.043}{\volt})(\SI{12.253}{\milli\ampere})\\
    P_{B0}&=\SI{49.539}{\milli\watt}
\end{align*}
In $P_{BC}$:
\begin{align*}
    P_{BC}&=V_{BC}I_2\\
    P_{BC}&=(\SI{10.545}{\milli\volt})(\SI{10.545}{\milli\ampere})\\
    P_{BC}&=\SI{-1.634}{\milli\watt}
\end{align*}
In $P_{C0}$:
\begin{align*}
    P_{C0}&=V_{C0}I_2\\
    P_{C0}&=(\SI{5}{\ohm})(\SI{1.708}{\milli\ampere})\\
    P_{C0}&=\SI{8.540}{\milli\watt}
\end{align*}
Finally, the sum of all powers is:
\begin{gather*}
    \sum\nolimits_{P}=P_{A0}+P_{AB}+P_{B0}+P_{BC}+P_{C0}\\
    \sum\nolimits_{P}=\SI{-94.905}{\milli\watt}+\SI{52.261}{\milli\watt}+\SI{49.539}{\milli\watt}-\SI{1.634}{\milli\watt}+\SI{8.540}{\milli\watt}\\
    %check!!!!!!!!!!!!!!!!
    \sum\nolimits_{P}=0
\end{gather*}
\subsubsection{Measurements}
\begin{table}[H]
    \centering
    \begin{tabular}{|c|c|c|}\hline
        Measurements & Theoretical Value & Measured Value\\
        & (\si{\milli\ampere}) & (\si{\milli\ampere})\\\hline 
        Current $I_1$ (Left branch) & 0 & 0 \\\hline
        Current $I_2$ (Center branch) & 0 & 0 \\\hline
        Current $I_3$ (Right branch) & 0 & 0 \\\hline
    \end{tabular}
    \caption{Theoretical and measured current values}
    \label{table:2}
\end{table}
\begin{table}[H]
    \centering
    \resizebox{\textwidth}{!}{%
    \begin{tabular}{|c|c|c|c|c|c|c|}\hline
    Measurements & Theoretical & Measured & Theoretical & Measured & Measured &
    Absorb(A)/\\
        &  Value (\si{\volt}) &  Value (\si{\volt}) &  Power (\si{\milli\watt}) & Current
        (\si{\milli\ampere}) & Power (\si{\milli\watt}) & Supply(S)\\\hline
        Voltage $V_{0A}$ & 0 & 0 & 0 & 0 & 0 & 0 \\\hline
        Voltage $V_{AB}$ & 0 & 0 & 0 & 0 & 0 & 0 \\\hline
        Voltage $V_{B0}$ & 0 & 0 & 0 & 0 & 0 & 0 \\\hline
        Voltage $V_{BC}$ & 0 & 0 & 0 & 0 & 0 & 0 \\\hline
        Voltage $V_{C0}$ & 0 & 0 & 0 & 0 & 0 & 0 \\\hline
    \end{tabular}}
    \caption{Theoretical and measured voltage values}
    \label{table:3}
\end{table}
\subsubsection{sim1}
\section{Questions}
\section{Conclusions}
{\large Sabrina:}\\
\\[2ex]
{\large Salvador:}\\
\\[2ex]
{\large Sebastián:}\\
\end{document}
