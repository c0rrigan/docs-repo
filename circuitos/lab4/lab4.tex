\documentclass[a4paper]{article}
\usepackage[T1]{fontenc}
\usepackage[utf8]{inputenc}
\usepackage{tocloft,siunitx,amssymb,amsmath,graphicx,subcaption,float,pgf,pgfplots}
\usepackage[top=3cm,left=3cm,right=3cm]{geometry}
\graphicspath{{img/}}
\renewcommand\cftsecfont{\normalfont}
\renewcommand\cftsecpagefont{\normalfont}
\renewcommand{\cftsecleader}{\cftdotfill{\cftsecdotsep}}
\renewcommand\cftsecdotsep{\cftdot}
\renewcommand\cftsubsecdotsep{\cftdot}
\renewcommand\cftsubsubsecdotsep{\cftdot}
\title{Lab 2:Ohm's Law}
\author{
    Sebastián Nava López\\
    \and
    Ericka Sabrina Pensamiento R.\\
    \and
    Salvador Palos Gil
}
\captionsetup[subfigure]{justification=raggedright}
\begin{document}
\begin{titlepage}
    \centering
    {\Huge Instituto Politécnico Nacional}\\[3ex]
    {\huge Escuela Superior de Cómputo}\\[8ex]
    {\huge Fundamental Circuit Analysis}\\[12ex]
    {\Large Lab 4: Oscilloscope Usage}\\[20ex]
    {\Large Group: 1CV7 Team: 7 \\[8ex]
    Sebastian Nava López\\[4ex]
    Sabrina Erika Pensamiento Robledo\\[4ex]
    Salvador Palos Gil\\[18ex]
    }
    \large{Elaboration: April 10,2018\hspace{8em} Due date: April 17,2018}
\end{titlepage}
\tableofcontents
\newpage
\section{Introduction}
\newpage
\section{Development}
\subsection{Measurement of the test signal}
In the first part of the experiment we connected a pair of test leads to each of the outputs in the
oscilloscope, then we clipped the probes connected to the channel one (CH1) to the test output,
at this point we could see a flat and thin yellow line along the display, then, using the knobs to
adjust the relation between volts per division and seconds per division displayed in the grid until we could see,
according to the label in the oscilloscope, a clear square wave with the volts per division
set at \SI{5}{\volt} and the frequency at \SI{300}{\hertz}. Afterwards we proceed in the same
way, now using the leads connected in channel two (CH2).
%!!!Fotos Parte1
\subsection{Testing the signal generator}
In the second part of the experiment we connected the signal generator to the oscilloscope and
set the generator to a frequency of \SI{10}{\kilo\hertz} and a frequency of $10 V_{pp}$,
alternating between the sine, triangular and squared waveforms
\subsubsection*{Sine wave}
\begin{gather*}
    Amplitude\ V_{pp}: \SI{5}{\volt}\\
    Period(T): \SI{100}{\micro\second}\\
    Frequence(f): \SI{10}{\kilo\hertz}
\end{gather*}
Once with the triangular waveform, we set the offset of the signal generator to the maximum, wich
was $5 V$, changing the offset moved the signal above in relation to the center of the display, then, when
we set the offset in the negative maximum, the signal moved below the center of the diplay.
%Poner foto y valores de máximo y mínimo offset
\subsection{The oscilloscope as an X-Y plotter with DC currents}
Using the circuit shown in picture:

We changed the mode of the oscilloscope yo X-Y, then, with each channel in ground coupling, we set
the trace for each channel to match the point at the center of the display, later, we clipped the
test probes to certain points of the circuit as follows:
1)Positive of channel X to point A and negative of channel X to point C
2)Positive of channel Y to point B and negative of channel Y to point C
%Primera foto
3)Positive of channel X to point A and positive of channel Y to point B, the negatives of both
channels to point C.
%Segunda foto
4)Same connection but with channel Y inverted.
%Tercera foto
4)Positive of channel X to point B, positive of channel Y to point C, the negatives of both
channels to point A and Y channel inverted.
%Cuarta foto
\subsection{The oscilloscope as an X-Y plotter with AC currents}
In the last part of the experiment, we used the following circuit:

then, with the oscilloscope set with AC coupling and Y(t) mode, we clipped the test probes to
their respective points according to the diagram shown before. After adjusting the size of each
graph, we could clearly see the phase shift between the graphs.
%!!!!Foto de fases
We have that:
\begin{gather*}
    Volts/div = \SI{2}{\volt}\\
    Time/div = \SI{1}{\micro\second}
\end{gather*}
If phase shift($\phi$) is equal to:
\[\frac{(\ang{360})(a)}{T}\]
where $a$ is the time difference between the signals, therefore:
\[\phi=\frac{(\ang{360})(\SI{0.4}{\second})}{\SI{3.2}{\second}}\qquad\therefore\phi=\ang{45}\]
Next, we changed to $X-Y$
mode, after adjusting voltage per division in both channels, we could spot a slightly rotated ellipse.
%!!!!!!!!Insertar fotos de elipse y curvas desfasadas
We have that:
\begin{gather*}
    Volts/div = \SI{2}{\volt}\\
\end{gather*}
Phase shift($\phi$) is given by:
\[\phi=\pm\sin^{-1}\Big(\frac{B}{A}\Big)\]
where $A$ is the y-coordinate of the peak of the ellipse in the first quadrant, and $B$ is the
y-coordinate where the ellipse intersects the x-axis, therefore:
\[\phi=\pm\sin^{-1}\Big(\frac{1.5}{2.1}\Big)\qquad\therefore\phi=\ang{45.584}\]
\section{Questions}
\section{Conclusions}
{\large Sabrina:}\\
\\[2ex]
{\large Salvador:}\\
\\[2ex]
{\large Sebastián:}\\
\end{document}
