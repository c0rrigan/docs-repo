\documentclass[a4paper,12pt]{article}
\usepackage[spanish]{babel}
\usepackage[utf8]{inputenc}
\usepackage[T1]{fontenc}
\usepackage[tmargin=2cm]{geometry}
\newcommand{\titlehomework}[4]{\begin{center}\section*{#4}{\large #2}\\#1\\#3\\[2ex]\end{center}}
\renewcommand{\labelitemi}{$\bullet$}
\begin{document}
 \titlehomework{Sebastián Nava López}{Comunicación Oral y Escrita}{1CV10}{A los ingenieros.}
 A los ingenieros.


Aaron Benitez es un mexicano, dueño de Verse Technology, firma que desarrolla hardware y software,  Verse Business, una consultoría de negocios, también ha sido profesor invitado de universidades como la UNAM o el Tecnológico de Monterrey, así como un escritor y conferencista. Yo lo conocí porque encontré uno de sus artículos en Medium, una plataforma de blogging, el artículo se llamaba Una nueva raza de ingenieros, en él básicamente se dedicaba a plantear sus inconformidades, así como algunas soluciones ante estas con respecto a la educación que se le da a los ingenieros, este texto resonó en mi porque actualmente estoy estudiando la Ingeniería en Sistemas Computacionales(ISC), y como él, yo también tengo algunos problemas con educación impartida en mi escuela. Uno de los problemas que siempre he tenido desde que comencé a estudiar dentro del sistema del Instituto Politécnico Nacional, empezando en el CECyT 9, es que las materias de humanidades no tienen un papel importante en el desarrollo de los estudiantes de las áreas de físico-matemáticas, esto provoca que mucha gente se conforme con resolver problemas de mecánica o saber integrar funciones, ignorando totalmente las habilidades de comunicación, y mis expectativas no han cambiado mucho ahora que estoy estudiando la ingeniería, creo que el modelo va a seguir siendo bastante similar y esto simplemente va a confirmar la percepción que yo tengo, y que mucha gente tiene de que los ingenieros son solo herramientas para llegar a un objetivo, más que verdaderos agentes de cambio e innovación de gran impacto, en su artículo, Benítez menciona que si bién, los conocimientos correspondientes a las áreas teóricas de las ingenierías son necesarios, sería bueno que los estudiantes también estuvieran expuestos a otros campos como el derecho, la administración, la política, el arte, etc. Por otro lado, el autor pone sobre la mesa la idea de “resolver más problemas realistas en lugar de materias idealistas”, en lugar de ocupar a los estudiantes con resolver problemas como, en el caso de ISC, crear algoritmos de ordenamiento o la utilización de instrumentos de medición, sino resolver problemas relacionados con el desarrollo dentro de la industria, aprender a tratar proveedores, aprender a manejar los errores provocados por una mala estipulación de requerimientos en el caso del desarrollo de software, saber solucionar las complicaciones de último momento al querer lanzar un producto al mercado, coincido con el autor que este tipo de problemas son los que deberíamos aprender a resolver a lo largo de nuestra educación superior;tratar de resolver este tipo problemas nos podría enseñar también a trabajar en equipo, otra habilidad que el autor también hace énfasis en que se necesita, y es que actualmente el trabajo en equipo muchas veces se ve reducido a simplemente delegar responsabilidades, esperar que todos cumplan con su parte, juntar todo y entregar, dejando fuera la verdadera colaboración que nos permite aprender sobre las demás personas y nos ayuda a mejorar nuestras habilidades interpersonales. Finalmente el autor propone la figura de los profesores invitados, figuras con trayectorias notables, no solo en el campo de conocimiento de los estudiantes sino también de otras áreas como las artes o los negocios, esto con la finalidad de alimentar a los estudiantes con nuevas ideas, de forma que puedan tener un mayor panorama, no solo en su área de desarrollo, sino de todas las otras posibilidades que hay más allá de su área.


En el caso de mi carrera, ISC, existen 7 materias que componen la parte que conforma toda la educación de las áreas de comunicación, administración y negocios, sin embargo opino que los trabajos en equipo, los debates, las actividades que implican exponer en público, hacer textos, no deberían ser exclusivas de estas materias, sino que deberían estar integradas dentro de las estrategias de las unidades de aprendizaje de la carrera de ISC, y tampoco sería mala idea, dentro de todas las demás ingenierías.


Sin embargo, creo que la llegada de estos cambios a los planes de estudio de una escuela pública como lo es la Escuela Superior de Cómputo, pueden ser tardados y no podemos esperar a que esos cambios lleguen, nosotros como estudiantes debemos de comenzar a tomar iniciativas que nos lleven a obtener lo que la educación no nos esté dando, comenzar a colaborar con compañeros en la elaboración de proyectos independientes para explorar tecnologías y asistir a reuniones sobre temas especializados son algunas ideas que pueden ayudar a desarrollarnos en esas áreas que quizá nuestra educación no cubra, deberíamos también tratar de expandir nuestro conocimiento a más campos afuera del área de nuestra propia ingeniería, tratar de ser lo más interdisciplinarios posibles y finalmente, deberíamos aprovechar las habilidades que se desarrollan en las materias del área humanística como escribir o saber hablar en público. Con todo esto podremos ser ingenieros capaces de encarar los retos que la industria nos presente, ingenieros capaces de desarrollar ideas pero también exponerlas y venderlas, de manera que no solo seamos un medio para llegar a objetivos, sino también establecer los objetivos.
\end{document}
