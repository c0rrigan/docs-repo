\documentclass[12pt]{article}
\usepackage[spanish]{babel}
\usepackage[utf8]{inputenc}
\usepackage[T1]{fontenc}
\usepackage[tmargin=2cm]{geometry}
\newcommand{\titlehomework}[4]{\begin{center}{\large \textbf{Reseña:\\{#4}}}\\[2ex]{\large #2}\\#1\\#3\\[2ex]\end{center}}
\renewcommand{\labelitemi}{$\bullet$}
\thispagestyle{empty}
\begin{document}
\titlehomework{Sebastián Nava López}{Comunicación Oral y Escrita}{1CV10}{\textit{Las
batallas en el desierto}  de José Emilio Pacheco}
Las batallas en el desierto es una novela corta escrita por el difunto   escritor mexicano José
Emilio Pacheco, ganador del premio Cervantes de literatura en 2009. El libro en
cuestión es \textit{Las batallas en el desierto}, publicado por primera vez en 1981, la historia esta contada desde
la perspectiva de un joven de clase media de la Ciudad de  México llamado Carlos, la historia nos sugiere que el personaje se encuentra en un momento de su vida en el cuál esta a punto de entrar a la adolescencia.
El la historia se desarrolla la Ciudad de México y sobretodo en la colonia Roma del el México post-revolucionario de la primera mitad
del siglo XX. La premisa de la historia es relativamente simple, trata sobre el amor prohibido que
tiene este joven hacía la madre de uno de sus amigos y al momento que este decide confesar su amor
a ella, el desenlace para todos termina siendo bastante trágico. Si bien la premisa no es compleja
durante todo el libro el autor nos lleva por un viaje a ese México donde Carlos vive, el México
de los los grandes presidentes, del predominante oficialismo en la prensa, el México donde mis abuelos se
enamoraron. Sin embargo, a pesar de que algunas cosas del pasado que menciona el autor ahora parecen
anticuadas, hay otras que aún podemos ver en el México del día de hoy como los presidentes jóvenes,
sonrientes y simpáticos, los amigos del presidente que hacen dinero a merced de los bienes del
estado, como alguien viviendo en el 2018 todo suena como lo primero que uno encuentra al entrar a
ver las noticias en cualquier medio de comunicación respetable, todo sigue siendo igual que en
1946, seguimos viviendo los remanentes de una revolución interrumpida, y como muchos de los
personajes de la novela, muchos mexicanos seguimos teniendo una especie de crisis de identidad que nos lleva a
inclinar nuestros gustos y basar nuestra identidad en el exterior, en lo extranjero, rechazando
nuestras raíces y despreciándolas como si fueran culturalmente inferiores, ese pasaje sigue resonando en nuestro
México del siglo XXI, cosa que es muy interesante considerando que la localización temporal del
libro nos coloca a 80 años de distancia con nuestros tiempos. Ese detalle que tiene el libro para
describir el contexto de la historia hace que, a pesar de ser un texto corto, el universo en el que
se desarrolla la historia tenga una mayor profundidad y, en mi caso, haya hecho que prestara mas
atención a lo secundario que a la historia del protagonista, sin embargo también es interesante ver
como se desarrolla el amor de este joven hacía una persona que no debería como es la madre de uno de
sus amigos, alguien que es notablemente mayor que él y sin embargo lo hace, se enamora de ella y lo
lleva tan lejos al punto de escaparse de la escuela y confesar su amor ,
a pesar de ser una decisión algo torpe dadas las probabilidades de que el resultado fuera positivo
para todos, personalmente creo que este tipo de amor que Carlos tiene hacía Mariana es una de las
formas mas puras de amor que una persona puede tener en su vida, ese amor que las personas solo
tienen a esa edad, cuando las preocupaciones y miedos del mundo exterior 
{\footnotesize (continua atras)}
\newpage 
simplemente no estan presentes y lo
único que importa es esa otra persona, no hay intereses ni obligaciones, ese aspecto del libro
también me agrado, ese amor visceral hacía Mariana. Finalmente la historia del libro me agrado por
ese amor visceral que tiene el protagonista hacia Mariana, sin embargo, la parte que más aprecié del libro fue
todo el universo que construye alrededor de la memoria de la Ciudad de México, y en general del
México de esos tiempos que nos permite reflexionar sobre la evolución,  pero también sobre
el estancamiento social, cultural y político que que vivimos como país y como sociedad mexicana en los últimos 80 años.
%remanentes de una revolución interrumpita
\end{document}

