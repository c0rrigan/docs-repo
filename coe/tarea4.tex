\documentclass[a4paper,12pt]{article}
\usepackage[spanish]{babel}
\usepackage[utf8]{inputenc}
\usepackage[T1]{fontenc}
\usepackage[tmargin=2cm]{geometry}
\usepackage{multicol}
\newcommand{\titlehomework}[4]{\begin{center}\section*{#4}{\large #2}\\#1\\#3\\[2ex]\end{center}}
\renewcommand{\labelitemi}{$\bullet$}
\renewcommand{\labelitemii}{$-$}
\thispagestyle{empty}
\begin{document}
\titlehomework{Sebastian Nava López}{Comunicación Oral y Escrita}{1CV10}{Uso
de la \textit{h}}
La letra \textit{h} se utiliza al principio de las palabras en los siguientes casos:
\begin{itemize}
    \item{Palabras que empiezan con \textit{hia-},\textit{hie-},\textit{hue-}, y
        \textit{hui-}.Ejemplos:
        \begin{multicols}{3}
             \begin{itemize}
              \item hueco
              \item hiato
              \item hueso
              \item huir
              \item hierro
              \item huila
        \end{itemize}
        \end{multicols}
        }
    \item{Palabras que empiezan con \textit{hum-},\textit{horm-} y \textit{horr-}.Ejemplos:
        \begin{multicols}{3}
             \begin{itemize}
              \item humedad
              \item humanidad
              \item horma
              \item horrible
              \item humano
              \item horrendo
        \end{itemize}
        \end{multicols}
        }
    \item{Palabras que empiezan con \textit{herm-},\textit{hern-},\textit{hog-} y
        \textit{holg-}.Ejemplos:
        \begin{multicols}{3}
             \begin{itemize}
              \item hermano
              \item hogar
              \item holgado
              \item hernia
              \item hermoso
              \item holgado
            \end{itemize}
        \end{multicols}
        }
\end{itemize}
Por otro lado, la letra \textit{h} se usa en las siguientes situaciones:
\begin{itemize}
    \item{Se escriben con \textit{h} las palabras que contienen la combinación
        \textit{vocal+ie}. La \textit{h} se coloca entre la vocal y el diptongo.Ejemplos:
        \begin{multicols}{3}
             \begin{itemize}
              \item ahuehuete
              \item alcahuete
              \item tehuelche
              \item azotehuela
              \item cacahuete
              \item colihue
        \end{itemize}
        \end{multicols}
        }
    \item{Palabras que empiezan con \textit{hum-},\textit{horm-} y \textit{horr-}.Ejemplos:
        \begin{multicols}{3}
             \begin{itemize}
              \item humedad
              \item humanidad
              \item horma
              \item horrible
              \item humano
              \item horrendo
        \end{itemize}
        \end{multicols}
        }
\end{itemize}
La letra \textit{h} también puede ser utilizada con la letra \textit{c} para formar el dígrafo
\textit{ch} que forma un solo sonido.\textit{ch} solo puede ser usado al principio de una
palabra o entre vocales.Ejemplo:
        \begin{multicols}{3}
             \begin{itemize}
              \item chopo
              \item cheque
              \item chasis
              \item derrochar
              \item lancha
              \item concha
        \end{itemize}
        \end{multicols}
\section*{Bibliografía}
\textit{Ortografía de la Lengua Española}(2010).Barcelona:Larousse Editorial,S.L. 
\end{document}
