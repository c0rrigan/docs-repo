\documentclass[a4paper,12pt]{article}
\usepackage[spanish]{babel}
\usepackage[utf8]{inputenc}
\usepackage[T1]{fontenc}
\usepackage[tmargin=2cm]{geometry}
\newcommand{\titlehomework}[4]{\begin{center}\section*{#4}{\large #2}\\#1\\#3\\[2ex]\end{center}}
\renewcommand{\labelitemi}{$\bullet$}
\renewcommand{\labelitemii}{$-$}
%\renewcommand{\labelitemiii}{$\cdot$}
\thispagestyle{empty}
\begin{document}
\titlehomework{Sebastián Nava López}{Comunicación Oral y Escrita}{1CV10}{Uso de \textit{j,g,gu} y \textit{gü}}
\begin{itemize}
\item{Uso de la j
\begin{itemize}
\item{ Formas verbales que se escriben en \textit{j}
\begin{itemize}
\item Verbos que terminan en \textit{-jar}. \\
\phantom{3em}Ejemplos: empujar,mojar,bajar,pujar,cortejar
\item Algunas formas de los verbos \textit{traer} y \textit{decir}. \\
\phantom{3em}Ejemplos: trajiste,dijiste,trajeron,dijimos
\item Algunas formas de los verbos cuyos infinitivos acaban en \textit{-ducir}\\
\phantom{3em}Ejemplos: dedujimos,sedujo,condujiste.
\end{itemize}
}
\item Palabras que inician con \textit{aje-} y \textit{eje-}\\
\phantom{3em}Ejemplos: ejecutar,ajedrez,ejemplificar,ajustar,ejercicio
\item Palabras que terminan en \textit{-aje},\textit{-eje},\textit{-jería} y sus compuestos\\
\phantom{3em}Ejemplos: brujería,salvaje,reportaje,viaje,montaje,plumaje
\end{itemize}
}
\item{Uso de la \textit{g}
\begin{itemize}
\item Palabras que inician en \textit{gest-,gene-} o \textit{geni-}\\
\phantom{3em}Ejemplos:gestión,generación,génico,genoma
\item Palabras que inician en \textit{leg-}\\
\phantom{3em}Ejemplos:legión,legendario,legado,legumbre
\item Palabras que terminan en \textit{-gen,-gélico,-gético,-genario,-génico,-géneo,-genio,-gésimo,-gesimal,-génito}\\
\phantom{3em}Ejemplos:congénito,evangélico,imagen,heterogéneo
\item Palabras que terminan en \textit{-gente} y \textit{-gencia}\\
\phantom{3em}Ejemplos:contingente,inteligencia,tangente,agencia.
\item Palabras que terminan en \textit{-gia,-gio,-gión,-gional,-ginal,-gionario,-gioso,-gírico}\\
\phantom{3em}Ejemplos:regional,prodigioso,original,panegírico.
\item Palabras que terminan en \textit{-ígena,-ígeno-,ígero}\\
\phantom{3em}Ejemplos:indígena,flamígero,oxígeno,tusígeno
\item Las terminaciones verbales \textit{-igerar,-ger-}y \textit{-gir}, y sus derivados\\
\phantom{3em}Ejemplos:aligerar,dirigir,converger,urgir.
\end{itemize}
}
\item{Uso de \textit{gu}
\begin{itemize}
\item El dígrafo \textit{gu}: Se utiliza cuando un par de letras presentan un solo sonido `g'. Ejemplos: conseguir,distinguir,aguerrido,burguesía.
\end{itemize}
}
\item{Uso de \textit{gü}
\begin{itemize}
\item Se utiliza una \textit{u} con diéresis cuando se combina el dígrafo \textit{gu} con \textit{i} o \textit{e} y suena la letra \textit{u}. Ejemplos:agüero,lingüista,cigüeña,ambigüedad,vergüenza
\end{itemize}
}
\end{itemize}
\end{document}
