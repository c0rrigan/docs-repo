\documentclass[a4paper,12pt]{article}
\usepackage[spanish]{babel}
\usepackage[utf8]{inputenc}
\usepackage[T1]{fontenc}
\usepackage[tmargin=2cm]{geometry}
\newcommand{\titlehomework}[4]{\begin{center}\section*{#4}{\large #2}\\#1\\#3\\[2ex]\end{center}}
\renewcommand{\labelitemi}{$\bullet$}
\thispagestyle{empty}
\begin{document}
\titlehomework{Sebastián Nava López}{Comunicación Oral y
Escrita}{1CV10}{Discurso: Nuestra generación}
Nuestra generación ha vivido algunos de los cambios sociales, políticos y económicos más
importantes en los últimos años.

Nuestra generación se gesto casi a la par con la alternancia de gobierno que tuvo este país a
finales del siglo pasado,por consiguiente, nosotros, a diferencia de nuestros antepasados, hemos vivido la mayor parte de nuestras vidas, en un estado que si bien, aún no podemos decir que es completamente democrático, es bastante diferente al que tuvieron que vivir nuestros padres y abuelos.

Los medios de comunicación tampoco son los mismos con los que convivieron las generaciones pasadas, el internet como medio de comunicación masiva también creció con nosotros. Hace 13 años, en 2005, la OCDE (Organización para la Cooperación y el Desarrollo Económicos), reportó que en
México, el porcentaje de hogares conectados a internet era del 8.9\%, en contraste, en la última
encuesta realizada por el INEGI, el porcentaje de hogares conectados era de 50.9\%

Conforme crecimos comenzamos a incluir cada vez más a la red en nuestras vidas, sobretodo en nuestra adolescencia. Probablemente muchos de nosotros, como yo, a través de descubrir nuevas cosas con nuestras computadoras logramos forjar algo que muchos de nuestros predecesores nunca lograron hacer: crear una identidad única fuera de lo establecido por los medios de comunicación populares, antiguos. Algunos voltearon a ver la cultura asiática, otros se dejaron deleitar por la literatura de terror inglesa y bueno, también hubo algunos que prefirieron quedarse en el umbral de la cultura popular.

Nuestra forma de comunicarnos fueron y aún lo son, las conversaciones por chat, las redes sociales como Facebook, Twitter.

A nosotros también nos ha tocado crecer con los nuevos gigantes tecnológicos y nos ha tocado ver cómo las grandes corporaciones del siglo XX decaen o simplemente mueren

General Electric, símbolo del \textit{boom} económico norteamericano posterior a la segunda guerra mundial, ahora sufre de un futuro incierto, este no es el caso de Facebook Inc, una empresa que con apenas 14 años de existencia, vende acciones a 200 dólares, 13 veces más caras que las acciones de General Electric en la actualidad.

Gracias al internet, podemos acceder a miles de años de generación de conocimiento, ya sea por cursos en línea de las mejores universidades del mundo, charlas por las mentes más brillantes de nuestros tiempos, libros digitalizados, etc.

Vaya tiempo en el que vivimos y sin embargo, a pesar de toda esta bonanza moderna, nuestra generación tiene un futuro bastante oscuro, uno que ni la generación de nuestros padres ni nuestros
abuelos esperaban. 

Cada día el capitalismo toma formas más agresivas hacia la sociedad, crea desigualdad social, controla las democracias y vuelve más estrecho el camino de la movilidad social.

Según el periodico The Guardian, el año pasado fue el más peligroso para los activistas dedicados a proteger el medio ambiente, con 4 activistas muertos cada semana, esto mientras los niveles del más siguen creciendo, estableciendo una tendencia que los especialistas dicen, es irreversible a este punto.

Y bueno, en nuestro país el panorama sigue siendo bastante oscuro, vivimos en un país lleno de fosas, sangre, corrupción; vacío de justicia y honestidad, un país sediento de esperanza.

Y en este momento es donde me pregunto, ¿qué puede hacer esta generación ante todo esto?

Nosotros tenemos a nuestra disposición muchas herramientas que los jovenes del movimiento del 68' no
tuvieron, la comunicación y la facilidad para documentar los hechos, y aún así 

Sería interesante ver de qué manera nuestra generación puede usar todas estas ventajas tecnológicas para superar los retos del mundo actual.

Como verán, en los últimos años el mundo ha cambiado frente a nosotros, ahora es nuestro turno de cambiar al mundo.
\end{document}
