\documentclass[a4paper,12pt]{article}
\usepackage[spanish]{babel}
\usepackage[utf8]{inputenc}
\usepackage[T1]{fontenc}
\usepackage[tmargin=2cm]{geometry}
\newcommand{\titlehomework}[4]{\begin{center}\section*{#4}{\large #2}\\#1\\#3\\[2ex]\end{center}}
\renewcommand{\labelitemi}{$\bullet$}
\thispagestyle{empty}
\begin{document}
\titlehomework{Sebastian Nava López}{Comunicación Oral y Escrita}{1CV10}{Reseña: Los 4 Acuerdos de
Miguel Ruiz}
%INTRODUCCIÓN
\section*{El Libro}
La primera parte de este libro nos da un poco de contexto sobre la concepción del
universo que tenían los toltecas y comienza a hablar sobre el \textit{sueño del planeta} y los
demás sueños que tenemos los seres humanos.Posteriormente habla sobre nuestro proceso de
formación en nuestra edad temprana y como se va formando nuestro \textit{libro de la
ley} con el que nos guiamos y define lo que nos da seguridad. En seguida nos dice que los
acuerdos son las ideas que están interiorizadas en nuestra mente y que nos definen como personas,
sin embargo puede haber acuerdos que actúen detrimento de nosotros.
Todo esto es producto de la \textit{domesticación}a la que se nos ha sometido desde pequeños, ya que el
mundo exterior siempre nos llevo a seguir sus reglas, seguir el \textit{sueño del planeta}.
Este seria en breve el resumen de la primera parte del libro sin embargo creo que lo
interesante son los cuatro acuerdos que bien dan el título al libro.
%En la época del México prehispanico existieron una grán cantidad de culturas que crearon su
%propia cosmogonía, su comida,etc.Una de ellas es la cultura Tolteca,
%El libro comienza proponiendonos su conepción del mundo y de lo que este esta hecho, en este
%caso, todo esta hecho de luz y esta luz crea estrellas, además, nosotros nos estamos en medio de
%las estrellas. Las estrellas son el tonal y la luz que existe entre ellas es el nagual y lo que
%crea la armonía entre ellas es la vida o intento, la vida siendo la fuerza de lo absoluto, la
%fuerza creadora de las cosas. De esta manera, todo la materia que vemos es producto de la
%reflexión de la luz y nuestra interpretación de esos reflejos, sin embargo también existee el
%sueño, el mundo de la ilusión que no permite ver lo que uno realmente es.
%Cuando este hombre descubrio esto se lleno de entusiasmo y fue a contarle todo a los demás, sin
%embargo pensaron que el era el Dios, ignorando que todos eran iguales y que ellos ambién eran
%este Dios. El hombre podía verse reflejado en ellos pero ellos no podían hacerlo ellos mismos ya
%que el sueño, ese muro de niebla que interpreta la luz, no los dejaba.\\
%\textbf{Domesticación y el sueño del planeta}\\
%En seguida el libro nos habla de que existe un \textit{sueño del planeta} del cuál forman parte
%otros sueños mas pequeños como el sueño de una familia, el sueño de una ciudad, el sueño de una
%nación y el sueño de la humanidad.A un ser humano desde pequeño le es transmitido el sueño
%colectivo del que nuestros críadores también forman parte, llamando nuestra atención, siendo
%recompensados cuando lo seguimos y castigados cuando no es de esa mnera, como si fueramos
%animales siendo domesticados. Nosotros nunca escogimos las creencias de ese sueño pero estabamos
%\textit{de acuerdo} ya que tenemos fe hacia los adultos. Con el tiempo vamos asumiendo esas reglas y las
%vamos poniendo en nuestro propio \textit{libro de la ley}, en el cuál dictara lo que es verdad y
%lo que nos da seguridad.Siempre que nos juzgamos utilzamos nuestro \textit{libro de la ley}
%y al ser juzgados, nosotros mismos somos la victima de estos jucios creados por el sistema de
%creencias que hemos sido inculcados y que a pesar de haber llegado a una adultez, ya ha sido
%interiorizado por nosotros.
%Nuestro sueño personal esta formado por conceptos que tenemos de nosotros mismos,los acuerdos a los que hemos llegado con los demás,
%nosotros mismo, etc, este sueño que es la bruma en la que vivimos, es lo que los toltecas
%llamaban \textit{mitote}, todas las personas hablan y nadie entiende. Esta bruma no nos deja ser
%nosotros mismos porque siempre hay que superar las expectativas de alguien mas, siempre hay que
%alinear nuestra visión con los demás y esto hace que una mismo se haga daño por juzgarnos y una
%vez que permitimos eso.\\
%Los acuerdos sobre lo que somos, como nos sentimos y que creemos forman nuestra personalidad,
%todos estos acuerdos definen de que manera vivimos, hay algunos buenos y algunos malos, nosotros
%podemos cambiar acuerdos por unos que puedan mejorar nuestra vida y además romper los otros
%acuerdos que no nos ayudan.\\[4ex] 


Existen Cuatro Acuerdos que el libro sugiere para ayudar a
transformar nuestra vida.
El primer acuerdo se llama  \textit{Ser impecable con tus palabras}, las palabras tienen el poder
de transmitir ideas hacia otras personas así como a nosotros mismos y estas palabras
pueden influir a los receptores de manera que incluso puedan afectar su sistema de creencias y en
dado caso, incluso nuestro mismo sistema de creencias.Podemos dibujar un paralelismo entre las
palabras y un campo fértil(nuestra mente), y un montón de semillas que pueden crecer en el(
las ideas).Las mentiras son un tipo de palabras que
,según autor, crean magia negra,envenenan y hechizan, y que solo pueden ser revertidas
con la verdad.La finalidad de ser impecable con las palabras es que nuestra mente se vuelva un
campo fértil de palabras de amor mas que de palabras venenosas.Este sería el acuerdo mas
importante y poderoso porque potencialmente puede empezar a destruir algunos acuerdos que impiden
a uno tener relaciones personales que transmitan amor.
El segundo acuerdo se llama \textit{No te tomes nada personalmente}, nos habla básicamente de que
las palabras del exterior o del mismo interior no deben afectarnos personalmente porque las
personas no conocen todo lo que somos por lo que es mejor dejar pasar ese \textit{veneno}. La causa de
que la mayor parte del tiempo tomemos las palabras de los demás de manera personal es que desde
pequeños se nos enseño a que todo lo que se nos decía siempre era culpa nuestra y solamente
nuestra,ignorando  que a veces, si no es que la mayor parte del tiempo, los demás o los acuerdos en nuestra cabeza se contrarían o
simplemente no conocen nuestra condición.
El siguiente acuerdo, es \textit{No hacer suposiciones}, así como lo dice el nombre mismo, no hay
que hacer suposiciones de nosotros sobre lo que hacen o piensan los demás ya que al hacerlo, la
persona puede reaccionar con \textit{veneno emocional} para nosotros o para ellos mismos. El suponer lo
que otros piensan nos puede llevar a la conclusión de que cierta persona o personas nos rechazaran
porque suponemos como piensan y predecimos que nos excluirán, sin embargo en un principio ni siquiera se pregunto para saber que tales afirmaciones
fueran ciertas.El objetivo de no hacer suposiciones es poder preguntar para tener las cosas
claras y hablar sin suposiciones, de manera que podamos comunicarnos clara y libremente
Finalmente, el último acuerdo es \textit{Haz siempre lo máximo que puedas}, nos dice que siempre
hay que hacer lo máximo que uno pueda,no el máximo físico donde uno se agota e incluso se
hace daño, tampoco dejar de esforzarse al grado de ser poco ambiciosos ya que nos podría dejar sentimientos de
frustración y culpas, sino mas bien siempre actuar por el bien nuestro, por el amor a lo que
hacemos. Este acuerdo engloba a los otros acuerdos ya que quizá no siempre se pueda ser impecable
con las palabras por el mismo ritmo de vida que tenemos, pero siempre se puede buscar el máximo
para hacerlo posible, quizá es imposible nunca volver a tomar las cosas personalmente pero se puede
intentar hacer el máximo para que no suceda, lo mismo pasa con dejar de hacer suposiciones. A
pesar de que a veces no se pueda cumplir los otros tres acuerdos anteriores en todo momento, cuando uno siempre
busca hacer el máximo se siente bien consigo mismo a pesar de eso.
Después de profundizar en los cuatro acuerdos, personalmente pienso que las siguientes secciones
del libro sirven mas como un complemento en el que profundizan algunos temas tocados en el libro
hasta ahora, mas que una parte fundamental del texto en este punto. En las siguiente parte
denominada \textit{El camino hacia la libertad}, ahonda en la formas en las que podemos romper con
los acuerdos que nos dañan, habla sobre la libertad y define lo que en realidad es y
sobre el parásito que no nos permite ser completamente libres. También ahonda en el tema de las
emociones y como es que pueden controlarnos y dañarnos si dejamos que lo hagan.
%Opinion
\section*{Mi opinión}
Debo decir que cuando comencé a investigar sobre el libro y me di cuenta de que clase de libro era comencé a
sospechar sobre la calidad del libro, pensaba que solo serían una bola de párrafos inverosímiles
creados para hacerle creer mentiras a las personas y vender millones en el camino. Este libro es
un best-seller, y debe ser juzgado como tal por lo que al leerlo no estaba esperando muchas
sorpresas, sin embargo logro captar mi
atención por la honestidad que cada uno de los acuerdos
tiene y la franqueza con la cuál el autor habla sobre ellos, creo que son guías que personalmente
creo que le harían bien a cualquier persona, lo único que requieren de nosotros es estar dispuestos
al cambio, estar dispuestos a buscar la felicidad aunque el camino a esta sea doloroso, sin embargo
es un camino que vale la pena empezar, y creo que las guías que nos da este libro nos serían de
mucha ayuda en tal recorrido. 
%yo estaba esperando un libro que se
%sintiera artifical, que me quisiera vender falsas esperanzas como muchos best.sellers de ayuda
%personal, sin embargo creo que fuí gratamente sorprendido.
Uno de los aspectos que más me agrada del libro es la humanidad con la que el autor
habla sobre los problemas, acepta que somos personas imperfectas y que siempre lo seremos pero lo
único que importa es que podamos vivir bien con nosotros mismos y con los demás, seguir siendo
auténticos y andar sin miedo en la vida. 
%semiconclusión
A pesar de que en un inicio pensaba que iba a odiar el libro ahora puedo decir que incluso se lo
recomendaría a alguien que quiero porque nadie debería de dejar que las palabras de las
personas los hieran, nadie debería de ser lastimado por hacer falsas suposiciones, nadie debería vivir sintiendo
que no esta haciendo lo mejor en este mundo y este libro nos ayuda a dejar eso, a empezara cambiar
para buscar la felicidad siendo auténticos y libres.
\end{document}
