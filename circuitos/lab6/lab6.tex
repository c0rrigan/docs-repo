\documentclass[letterpaper]{article}
\usepackage[T1]{fontenc}
\usepackage[utf8]{inputenc}
\usepackage{tocloft,siunitx,amssymb,amsmath,graphicx,float}
\usepackage[top=3cm,left=3cm,right=3cm]{geometry}
%\usepackage{float,pgf,pgfplots}
%\usepackage[american]{circuitikz}
\graphicspath{{img/}}
\renewcommand\cftsecfont{\normalfont}
\renewcommand\cftsecpagefont{\normalfont}
\renewcommand{\cftsecleader}{\cftdotfill{\cftsecdotsep}}
\renewcommand\cftsecdotsep{\cftdot}
\renewcommand\cftsubsecdotsep{\cftdot}
\renewcommand\cftsubsubsecdotsep{\cftdot}
\title{Lab 6:Nodal Analysis}
\author{
    Sebastián Nava López\\
    \and
    Ericka Sabrina Pensamiento R.\\
    \and
    Salvador Palos Gil
}
%\captionsetup[subfigure]{justification=raggedright}
\begin{document}
\begin{titlepage}
    \centering
    {\Huge Instituto Politécnico Nacional}\\[3ex]
    {\huge Escuela Superior de Cómputo}\\[8ex]
    {\huge Fundamental Circuit Analysis}\\[12ex]
    {\Large Lab 6:Nodal Analysis(DC)}\\[20ex]
    {\Large Group: 1CV5 Team: 7 \\[8ex]
    Sebastian Nava López\\[4ex]
    Sabrina Erika Pensamiento Robledo\\[4ex]
    Salvador Palos Gil\\[18ex]
    }
    \large{Elaboration:April 24, 2018 \hspace{8em} Due date:May 1, 2018}
\end{titlepage}
\tableofcontents
\newpage
\section{Introduction}
\newpage
\section{Development}
\subsection{Verification of nodal analysis in DC}
We wired a circuit as shown in the diagram in figure %ref
%circuito
then, we measured voltage values in each node and currents in each resistor and each voltage
source. Then, measuring the voltage in each resistor, we calculated power values for each
component.
\subsubsection{Measurements}
First, in order to use nodal analysis to find the voltages in the nodes of our circuit, first we
need to declare the circuit currents, in this case, we do it as shown in figure %fig:
%diagrama2
Also, looking in the circuit, we note that our circuit has a supernode going between $V_2$ and
$V_3$, therefore:
\[V_2-V_3=\SI{5}{\volt}\]
\begin{equation}
    V_2 = \SI{5}{\volt}+V_3
    \label{eq:1}
\end{equation}
Then, applying \textit{KCL} in the supernode:
\[I_1+I_2=1_2+I_4\]
Using Ohm's Law to express the currents:
\begin{gather*}
    \frac{V_3-V_1}{\SI{330}{\ohm}}-\frac{V_2}{\SI{560}{\ohm}}=\frac{V_1-V_2}{\SI{1}{\kilo\ohm}}-\frac{V_3}{\SI{680}{\ohm}}\\
    \frac{1}{10}\Bigg[\frac{V_3-V_1}{\SI{330}{\ohm}}-\frac{V_2}{\SI{560}{\ohm}}=\frac{V_1-V_2}{\SI{1}{\kilo\ohm}}-\frac{V_3}{\SI{680}{\ohm}}\Bigg]
\end{gather*}
\begin{equation}
    \frac{V_3-V_1}{33}-\frac{V_2}{56}=\frac{V_1-V_2}{100}-\frac{V_3}{68}
    \label{eq:2}
\end{equation}
Also, we know that voltage in node $V_1$ is given by:
\begin{equation}
    V_1=\SI{9}{\volt}
    \label{eq:3}
\end{equation}
Using \eqref{eq:1} and \eqref{eq:3} in \eqref{eq:2}:
\begin{gather*}
    \frac{V_3}{33}-\frac{9}{33}+\frac{5+V_3}{56}=\frac{9}{100}-\frac{5+V_3}{100}\\
    -\frac{9}{33}+\frac{5}{56}+\frac{5}{100}-\frac{9}{100}=-\frac{V_3}{100}-\frac{V_3}{56}-\frac{V_3}{100}-\frac{V_3}{68}\\
    -\frac{3441}{15400}=\frac{-(56)(100)(68)V_3-(33)(100)(68)V_3-(33)(56)(68)V_3-(33)(56)(100)V_3}{(100)(33)(56)(68)}\\
    -\frac{3441}{15400}=-\frac{915664V_3}{(100)(33)(56)(68)}\\
    V_3=\frac{(3441)(100)(33)(56)(68)}{(-915664)(15400)}\\
    \therefore V_3=\SI{3.066}{\volt}
\end{gather*}
By using \eqref{eq:1}, $V_2$ is given by:
\begin{gather*}
    V_2=\SI{5}{\volt}+\SI{3.066}{\volt}\\
    \therefore V_2 = \SI{8.066}{\volt}
\end{gather*}
The currents in the resistors are, for $R_1$:
\begin{gather*}
    I_{R1}=\frac{V_3-V_1}{R_1}=\frac{3.066-9}{\SI{330}{\ohm}}\\
    \therefore I_{R1}=\SI{-17.982}{\milli\ampere}
\end{gather*}
For $R_2$:
\begin{gather*}
    I_{R1}=\frac{V_1-V_2}{R_2}=\frac{9-8.066}{\SI{1}{\kilo\ohm}}\\
    \therefore I_{R1}=\SI{934}{\micro\ampere}
\end{gather*}
For $R_3$:
\begin{gather*}
    I_{R3}=\frac{V_2}{R_3}=\frac{8.066}{\SI{560}{\ohm}}\\
    \therefore I_{R3}=\SI{14.403}{\milli\ampere}
\end{gather*}
For $R_4$:
\begin{gather*}
    I_{R4}=\frac{-V_3}{R_4}=\frac{-3.066}{\SI{680}{\ohm}}\\
    \therefore I_{R4}=\SI{-4.509}{\milli\ampere}
\end{gather*}
Current values for each current source are given as follows. For $V_{S1}$:
\begin{gather*}
    V_{S1} = I_{R_1}-I_{R_2} =
    \SI{-17.982}{\milli\ampere}-\SI{934}{\micro\ampere}\\
    \therefore V_{S1} = \SI{-18.916}{\milli\ampere}
\end{gather*}
For $V_{S2}$:
\begin{gather*}
    V_{S2} = I_{R_1}+I_{R_4} =
    \SI{-17.982}{\milli\ampere}-(\SI{-4.509}{\milli\ampere})\\
    \therefore V_{S1} = \SI{-13.473}{\milli\ampere}
\end{gather*}
Power values for each component in the circuit are given as follows, for $R_1$:
\begin{figure}[H]
    \centering
    \begin{tabular}{|c|c|c|c|}
        \hline
        Measurements & Theoretical Value($\si{\volt}$) & Measured Value($\si{\volt}$) & Simulated
        Value($\si{\volt}$)\\\hline
        $V_1$ & 9.000 & 9.120 &9.000 \\\hline
        $V_2$ & 8.066 & 8.160 & 8.066 \\\hline
        $V_3$ & 3.066 & 3.083 &3.066 \\\hline
    \end{tabular}
    \caption{Calculated, measured and theoretical voltage values for each node}
\end{figure}
\begin{figure}[H]
    \centering
    \begin{tabular}{|c|c|c|c|}
        \hline
        Measurements & Theoretical Value($\si{\milli\ampere}$) & Measured
        Value($\si{\milli\ampere}$) & Simulated Value($\si{\milli\ampere}$)\\\hline
        $I_{R1}$ & -17.982 & -17.780 & -17.980\\\hline
        $I_{R2}$ & 0.934 & 0.940 & 0.871\\\hline
        $I_{R3}$ & 14.403 & 14.408& 14.404\\\hline
        $I_{R4}$ & -4.509 & 4.550 & 4.509\\\hline
        $I_{S1}$ & -18.916& -18.240& -18.914\\\hline
        $I_{S2}$ & -13.473 & -14.050& -13.471\\\hline
    \end{tabular}
    \caption{Calculated, measured and theoretical current values}
\end{figure}
\begin{figure}[H]
    \centering
    \begin{tabular}{|c|c|c|c|c|}
        \hline
    Measurements & Theoretical & Measured & Simulated & 
    Absorb(A)/\\
     &  Power (\si{\watt}) &  Power (\si{\watt})  & Power (\si{\watt}) & Supply(S)\\\hline
        $R_1$ & & 0.108 & 0.106 &A\\\hline
        $R_2$ & & \num{8.94e-4}& \num{8.71e-4}&A\\\hline
        $R_3$ & & 0.117 & 0.116 &A\\\hline
        $R_4$ & & 0.013 & 0.014&A\\\hline
        $V_{S1}$ & &-0.164 & -0.170&S\\\hline
        $V_{S2}$ & & -0.067 & -0.067&S\\\hline
    \end{tabular}
    \caption{Calculated, measured and theoretical power values for each component}
\end{figure}
\section{Questions}
\textit{\textbf{Explain the operation of the oscilloscope}}\\
%ans
\textit{\textbf{What's the function of the signal generator?}}\\
%ans
\textit{\textbf{What are the Lissajous graphs for?}}\\
%ans
\textit{\textbf{What are the operating modes Y (t) and XY used for?}}\\
%ans
\textit{\textbf{Which means by coupling in D.C.?}}\\
%ans
\section{Conclusions}
{\large Sabrina:}\\
%
\\[2ex]
{\large Salvador:}\\
%
\\[2ex]
{\large Sebastián:}\\
%texto
\end{document}
