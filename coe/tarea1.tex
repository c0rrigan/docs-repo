\documentclass[a4paper,12pt]{article}
\usepackage[spanish]{babel}
\usepackage[utf8]{inputenc}
\usepackage[T1]{fontenc}
\usepackage[tmargin=2cm]{geometry}
\renewcommand{\labelitemi}{$\bullet$}
\thispagestyle{empty}
\begin{document}
\begin{center}
\section*{Comunicación:definiciones}
{\large Comunicación Oral y Escrita}\\
Sebastian Nava López\\
1CV10\\[2ex]
\end{center}
Paul Watzlawick et.al en su libro Teoría de la Comunicación Humana(1985) menciona que la comunicación es el vehículo en el que se mueven las interacciones entre el sistema que es la conducta humana, los efectos de esta ante los demás,la reacción de estos últimos a esta conducta y el contexto sobre el que tiene lugar

Por otro lado Frias(2000) nos dice que "La comunicación consiste, básicamente, en la transmisión de un mensaje de una persona o grupo a otro, lo que requiere de la existencia de voluntad de interacción entre ambas partes, es decir que se cree un proceso de influencia mutua y recíproca, mediante el intercambio de pensamientos, sentimientos y reacciones que se manifiestan a través del feed-back que se establece en los comunicantes", también establece las siguientes condiciones para que la comunicación sea posible:
\begin{itemize}
\item{Se produzca una relación entre los actores comunicantes(al menos en el momento puntual de las transmisiones}
\item{Se haga uso del mismo lenguaje}
\item{El receptor decodifique el mensaje recibido según su propio sistema de pensamiento,que se halla inscrito en el sistema de normas y valores del medio en el que opera}
\item{La respuesta estará en función de la comprensión y de los condicionantes individuales,organizacionales y sociales del receptor}
\end{itemize}
\section*{Bibliografía}
Frias Azcárate, R. (2000). \textit{Una aproximación al concepto comunicación y sus consecuencias en la práctica de las instituciones} Nómadas.Consultado de http:// www.redalyc.org/pdf/181/18100103.pdf\\
Watzlawick, P., Jackson, D. D., \& Bavelas, J. B. (1985). \textit{Teoría de la comunicación humana: interacciones, patologías y paradojas.} Barcelona: Editorial Herder. 
\end{document}