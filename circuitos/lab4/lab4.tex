\documentclass[a4paper]{article}
\usepackage[T1]{fontenc}
\usepackage[utf8]{inputenc}
\usepackage{tocloft,siunitx,amssymb,amsmath,graphicx,subcaption,float,pgf,pgfplots}
\usepackage[top=3cm,left=3cm,right=3cm]{geometry}
\graphicspath{{img/}}
\renewcommand\cftsecfont{\normalfont}
\renewcommand\cftsecpagefont{\normalfont}
\renewcommand{\cftsecleader}{\cftdotfill{\cftsecdotsep}}
\renewcommand\cftsecdotsep{\cftdot}
\renewcommand\cftsubsecdotsep{\cftdot}
\renewcommand\cftsubsubsecdotsep{\cftdot}
\title{Lab 2:Ohm's Law}
\author{
    Sebastián Nava López\\
    \and
    Ericka Sabrina Pensamiento R.\\
    \and
    Salvador Palos Gil
}
\captionsetup[subfigure]{justification=raggedright}
\begin{document}
\begin{titlepage}
    \centering
    {\Huge Instituto Politécnico Nacional}\\[3ex]
    {\huge Escuela Superior de Cómputo}\\[8ex]
    {\huge Fundamental Circuit Analysis}\\[12ex]
    {\Large Lab 4: Oscilloscope Usage}\\[20ex]
    {\Large Group: 1CV7 Team: 7 \\[8ex]
    Sebastian Nava López\\[4ex]
    Sabrina Erika Pensamiento Robledo\\[4ex]
    Salvador Palos Gil\\[18ex]
    }
    \large{Elaboration: April 10,2018\hspace{8em} Due date: April 17,2018}
\end{titlepage}
\tableofcontents
\newpage
\section{Introduction}
\newpage
\section{Development}
\subsection{Measurement of the test signal}
In the first part of the experiment we connected a pair of test leads to each of the outputs in the
oscilloscope, then we clipped the probes connected to the channel one (CH1) to the test output,
at this point we could see a flat and thin yellow line along the display, then, using the knobs to
adjust the relation between volts per division and seconds per division displayed in the grid until we could see,
according to the label in the oscilloscope, a clear square wave with the volts per division
set at \SI{5}{\volt} and the frequency at \SI{300}{\hertz}. Afterwards we proceed in the same
way, now using the leads connected in channel two (CH2).
\subsubsection{Measurements}
\subsubsection{Simulations}
\subsection{Testing the signal generator}
In the second part of the experiment we connected the signal generator to the oscilloscope and
set the generator to a frequency of \SI{10}{\kilo\hertz} and a frequency of $10 V_{pp}$,
alternating between the sine, triangular and squared waveforms
\subsubsection{Measurements}
\subsubsection{Simulations}
Once with the triangular waveform, we set the offset of the signal generator to the maximum, wich
was $5 V$, the change in the offset moved the signal above the center of the display, then, when
we set the 

\section{Questions}
\section{Conclusions}
{\large Sabrina:}\\
\\[2ex]
{\large Salvador:}\\
\\[2ex]
{\large Sebastián:}\\
\end{document}
