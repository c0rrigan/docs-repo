\documentclass[a4paper,11pt]{article}
\usepackage[spanish]{babel}
\usepackage[utf8]{inputenc}
\usepackage[T1]{fontenc}
\usepackage[tmargin=1.3cm,bmargin=1cm]{geometry}
\newcommand{\titlehomework}[4]{\begin{center}\section*{#4}{\large #2}\\#1\\#3\\[1.5ex]\end{center}}
\renewcommand{\labelitemi}{$\bullet$}
\renewcommand{\labelitemii}{$-$}
\thispagestyle{empty}
\begin{document}
% \titlehomework{autor}{Materia}{grupo}{nombre de tarea}
\titlehomework{Sebastián Nava López}{Comunicación Oral y Escrita}{1CV10}{Niveles de Comunicación}
Agüero(2012) propone una clasificación general de los niveles de comunicación de la siguiente manera:
\begin{itemize}
\item Intrapersonal: Es el procesamiento individual de la información, sin embargo su validez como nivel de comunicación puede ser cuestionable por la ausencia de un segundo actor
\item Interpersonal o cara a cara: Es la comunicación entre 2 o más personas físicamente próximas
\item Intragrupal: Es la que se establece dentro de un grupo específico como la familia.
\item Intergrupal: Es la que se realiza en la comunidad local. A ella es inherente la comunicación pública, la cuál se produce y distribuye en un sistema de comunicación especializado y que concierne a la comunidad en conjunto.
\item Institucional u organizacional: Es la asignación de recursos humanos y materiales a una organización para la generación,obtención,procesamiento y distribución de información destinada para comunicación pública. Es la comunicación del sistema político y de las empresas comerciales.
\item Sociedad: Es la comunicación que tiene como fuente una organización formal y que se dirige a una gran audiencia mediante un comunicador profesional, estableciendo así una comunicación unidireccional
\end{itemize}
Por otro lado,según Maletzke(1963) podemos la comunicación en  los siguientes tipos con sus correspondientes niveles:
\begin{itemize}
\item Directa e indirecta
\begin{itemize}
\item Directa: Se realiza de manera inmediata, sin intermediarios	
\item Indirecta: Realizada por interlocutores separados por el tiempo o espacio o ambas simultáneamente e.g. comunicación telefónica,emisiones de televisión y radio.
\end{itemize}
\item Recíproca y unilateral
\begin{itemize}
\item Recíproca: Ambas partes pueden intercambiar sus papeles de emisor y receptor
\item Unilateral: No se intercambian los roles de los comunicadores
\end{itemize}
\item Privada y pública
\begin{itemize}
\item Privada: Se dirige exclusivamente a un grupo determinado de personas
\item Pública: El grupo de receptores no esta determinado ni delimitado, cualquiera que tenga acceso y/o interés a él podrá recibirlo
\end{itemize}
\item Con o sin retorno
\end{itemize}
\section*{Bibliografía}
Agüero, P. M. Z. (2012). \textit{La comunicación interpersonal} EUMED-Universidad de Málaga.\\
Maletzke, G. (1963). \textit{Psychologie der Massenkommunikation: Theorie und Systematik}Hamburgo: Hans-Bredow-Institut. 
\end{document}