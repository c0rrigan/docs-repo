\documentclass[12pt]{article}
\usepackage[spanish]{babel}
\usepackage[utf8]{inputenc}
\usepackage[T1]{fontenc}
\usepackage[tmargin=2cm]{geometry}
\newcommand{\titlehomework}[4]{\begin{center}{\large \textbf{Reseña:\\{#4}}}\\[2ex]{\large #2}\\#1\\#3\\[2ex]\end{center}}
\renewcommand{\labelitemi}{$\bullet$}
\thispagestyle{empty}
\begin{document}
\titlehomework{Sebastián Nava López}{Comunicación Oral y Escrita}{1CV10}{\textit{Las
batallas en el desierto}  de José Emilio Pacheco}
Las batallas en el desierto es una novela corta escríta por el relativamente reciente difunto Jose
Emilio Pacheco, escritor mexicano ganador del premio Cervantes de literatura en 2009. El libro en
cuestión, \textit{Las batallas en el desierto}, públicado por primera vez en 1981 esta contado desde
la perspectiva de un joven de clase media de la ciudad de  México llamado Carlos, que se encuentra punto de entrar a la adolesencia,
El contexto de la historia es la colonia Roma en el México post-revolucionario de la primera mitad
del siglo XX. La premisa de la historia es relativamente simple, trata sobre el amor prohibido que
tiene este joven hacía la madre de uno de sus amigos y al momento que este decide confesarle su amor
a ella el desenlace para todos termina siendo bastante trágico. Si bien la premisa no es compleja
durante todo el libro el autor nos lleva por un viaje a ese México donde Carlos vive, el México
de los los grandes presidentes, del predominante oficialismo en la prensa, el México donde mis abuelos se
enamoraron. Sin embargo, a pesar de que algunas cosas del pasado que meciona el autor ahora parecen
anticuadas, hay otras que aún podemos ver en el México del día de hoy como los presidentes jovenes,
sonrientes y simpáticos, los amigos del presidente que hacen dinero a merced de los bienes del
estado, como alguíen viviendo en el 2018 todo suena como lo primero que uno encuentra al entrar al
ver las noticias en cualquier medio de comunicación responsable, todo sigue siendo igual que en
1946, seguimos viviendo los remanentes de una revolución interrumpida, y como muchos de los
personajes de la novela, muchos mexicanos seguimos teniendo una crísis de identidad que nos lleva a
inclinarnos nuestros gustos y basar nuestra identidad en el exterior, a lo extranjero, rechazando
nuestras raices y despreciarlas como si fueran menos, ese otro pasaje sigue resonando en nuestro
México del siglo XXI, cosa que es muy interesante considerando que la localización temporal del
libro nos coloca a 80 años de distancia con nuestros tiempos. Ese detalle que tiene el libro para
describir el contexto de la historia hace que, a pesar de ser un texto corto, el universo en el que
se desarrolla la historia tenga una mayor profundidad y, en mi caso, halla hecho que prestara mas
atención a lo secundario que a la historia del protagonista, sin embargo también es interesante ver
como se desarrolla el amor de este joven hacía una persona que no debería como es la madre de uno de
tus iguales, alguíen que es notablemente mayor que él y sin embargo lo hace, se enamora de ella y lo
lleva tan lejos al punto de escaparse de la escuela y confesarselo a Mariana, la madre de su amigo,
a pesar de ser una decisión algo torpe dadas las probabilidades de que el resultado fuera positivo
para todos, personalmente creo que este tipo de amor que Carlos tiene hacía Mariana es una de las
formas mas puras de amor que una persona puede tener en su vida, ese amor que las personas solo
tienen a esa edad, cuando las preocupaciones y miedos del mundo exterior simplemente no existen y
{\footnotesize (continua atras)}
\newpage
único que importa es esa otra persona, no hay intereses ni obligaciones, ese aspecto del libro
también me agrado, ese amor visceral hacía Mariana. Finalmente la historia del libro me agrado por
el todo el tema del amor visceral de Carlos a Mariana pero la parte que más aprecié del libro fue
todo el universo que construye alrededor de la memoria de la Ciudad de México, y en general del
México de esos tiempos que nos permite reflexionar sobre la evolución pero también de alguna manera
el estancamiento que que hemos tenido como país y como sociedad mexicana en los últimos 80 años.
%remanentes de una revolución interrumpita
\end{document}

