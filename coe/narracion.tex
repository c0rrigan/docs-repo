\documentclass[letterpaper]{article}
\usepackage[spanish]{babel}
\usepackage[utf8]{inputenc}
\usepackage[T1]{fontenc}
\newcommand{\titlehomework}[4]{\begin{center}\section*{#4}{\large #2}\\#1\\#3\\[2ex]\end{center}}
\renewcommand{\labelitemi}{$\bullet$}
\thispagestyle{empty}
\begin{document}
% \titlehomework{autor}{Materia}{grupo}{nombreDeTarea}
19/09
Martes diecinueve de septiembre del 2017, me encontraba en mi cama descansando el agitado día
anterior, sin duda los lunes son los días mas terribles de la semana, la matería de Ética, Cálculo,
y finalmente Análisis Vectorial a la última hora, todo se junta en mi contra, lo único bueno es la
hora de Mates' Discretas con Olguita, la maestra rusa. Como todas las mañanas, despertaba, escuchaba
mi despertador, ingresaba el patrón para desactivarlo, regresaba a las cobijas sintiendome mal
porque no puedo seguir durmiendo, de nuevo tomo mi telefono y busco en Facebook las mismas fotos de
la misma persona que sigo extrañando desde hace 4 años, como siempre, me doy cuenta que debería
dejarlo, el sonido del motor del exprimidor de jugo me altera un poco, ya es hora de desayunar. Me
levanto, busco mis zapatos, llego a la cocina y ahí estan mis padres, los saludo antes de proceder a
consumir los alimentos.Como todas las mañanas, reviso las noticias en el The Guardian o en el sitio
de Carmen Aristegui, nada nuevo, Donald Trump sigue destruyendo todo lo que ve, %Aumentar
acabo de desayunar, son las 11 de la mañana, tengo que estar listo para salir a las 12:30 si quiero
llegar a tiempo a mi primera hora a la escuela, hoy toca Matemáticas Discretas a la primera hora,
aunque puedo llegar tarde, necesito comenzar a subir esa calificación, apenas he podido aprenderme
los silogismos lógicos y la clase ya esta en %el tema que va despues de silogismos
Se me va la mañana deslizandome infinitamente en mi muro de Facebook sin darme cuenta que ya eran
las 11:30, 11:48, mire la esquina del monitor y eran 12:03, ya iba tarde, tenía que correr, o
debería?, de cualquier forma solo estamos resolviendo problemas en clase, puedo retrasarme unos
minutos. Salí de bañarme, me despedí de todos en la casa, salí a la calle, cruce la avenida, iba
caminando rápidamente para llegar a la avenida Pantitlan, una vez ahí tome el camión al metro
Pantitlan, como siempre, subí con miedo, esperando a que en cualquier momento alguíen subiera a
pedir dinero de forma pasivo-agresiva, finalmente llegue al metro sin ninguna novedad.Subí al anden
de la línea 5 a esperar el tren, el número de personas esperando era el mismo de siempre, despúes de
2 minutos esperando el tren arribo, como siempre la gente anciosa de entrar abrio las puertas del
vagon y yo aproveche para alcanzar un lugar donde no alcanzara el sol, como de costumbre miré
vagamente hacía el pasajero que estaba frente a mi, hoy había una mujer joven , morena,su
vestimenta se encontraba en el punto entre ser informal y ser lo que vestirias en la oficina,
por la forma en la que miraba su teléfono y buscaba cosas en su grande bolso blanco probablemente tenía prisa,
pero en esta ciudad quién no lo esta, técnicamente yo debería de estar, deje de mirarla para que no
pensara que fuera una persona rara, o por lo menos que no tuviera razónes para justificarlo, ya hago
lo suficiente cargando con esta cara y este cuerpo.El tren, despúes de algunos segundos de avisar su partida,
cerro las puertas y comenzo a avanzar, procedí a sacar los audifonos de mi mochila, los conecte a mi
teléfono y comence a navegar entre los artistas que tenía guardados, a pesar de que el Verano ya se
iba, el sol era algo intenso y la temperatura no era precisamente cálida pero tampoco fría. Me
pregunto que debería escuchar para acompañar este viaje, ¿algo de Pink Floyd?, ¿música
\textit{ambient}?,quiza debería que dejar que la reproducción aleatoria lo decida, el tren comenzo a
decender hacía el tunel y la radiación del sol cedío, y al poco tiempo de bajar al tunél, el tren
dejo de circular, las luces se apagaron, probablemente otro desperfecto más de este horrible
servico pense...hasta que se empezo a mover la tierra, el vagon se balanceaba de forma suave pero
frecuente sobre los rieles, de pronto lo único que observaba a traves de la ventana era la pared del
tunel y lo que parecían pedazos de piedra y polvo que se desprendian de las paredes del túnel, en
las bocinas del tren el conductor alertaba sobre el movimiento telúrico que estaba tomando lugar en
ese momento, la
primera cosa que hice al darme cuenta de lo que estaba pasando era poner mi canción favorita en mi
teléfono, \textit{Echoes} de \textit{Pink Floyd} en su versión en vivo de la película \textit{Live
at Pompeii}, si mi existencía en este mundo se encontraba en duda a partir de este momento prefiero
dejar este mundo disfrutando una de las cosas que más me gustan, mientras hacía esto el terror y la
angustía habían tomando posesión de todas las peronas a mi alrededor, la situación empeoro
cuando un par de bebes que se encontraban en el vagón comenzaron a llorar, abonando a la histería
colectiva de todos los que nos encontrabamos en ese vagón. Despúes alrededor de 1 minuto la tierra
se calmo pero las emociones de los que nos encontrabamos ahí no, algunos querían abrir las puertas y
a la vez apelaban a tomar la calma y esperar a que llegara el personal del servicio del metro, en
mis audifonos seguía sonando la música, trataba de no ponerme nervioso, de cualquier forma no tenía
forma fácil de salir así que anguistiarse era inútil, sin embargo la chica que estaba sentada frente
a probablemente no pensaba lo mismo, estaba tratando de comunicarse con alguíen al telefono pero
evidentemente las líneas estaban saturadas y además estabamos bajo tierra, en el túnel, seguía
moviendose nerviosamente tratando ,sin tener éxito, de encontrar señal. Por un momento pense que
podría acercarme a la chica del asiento de frente y tratar de tranquilizarla, despues de todo era el
único que seguía sentado cerca de ella, todos se habían parado a caminar o a tratar de ver hacía
tunel por las puertas del vagón, era el momento pero al mismo momento \textit{yo} seguía siendo
\textit{yo}, otro programador más con habilidades de sociales tan buenas que despúes de dos meses en la escuela sigue
sin hablarle a nadíe en la escuela, nunca le hable, ella seguía sufriendo, la desesperación se había
apoderado de ella y no podía ayudarla. Pasaron diez minutos desde el temblor y los ánimos comenzaron
a calmarse, las personas lentamente comenzaban a tranquilizarse e incluso a hacer bromas entre
ellos, la chica del asiento de enfrete pudo prender la luz de la cámara de su teléfono, supongo que
la encontrarse en la oscuridad también la ponía nerviosa, yo por otro lado estaba pensando en cómo
estarán las cosas alla afuera, el temblor había sido algo fuerte pero quiza el movimiento fue de
alguna forma amplificado por estar en un túnel bájo tierra, pensaba en mi familia, me preguntaba
dónde estarían ellos, sobre mi amigo Uriel. Paso media hora y el auxilio no aparecía, pasaron 40
minutos y aún seguiamos en la oscuridad, de repente una luz que provenía del vagon anterior al
nuestro llamo la atención de todas las personas, finalmente habían llegado a sacarnos, una
trabajadora del metro abrio la puerta que se encuentra en el extremo de los vagones para poder
entrar, cuando entró pidio la ayuda de un par de hombres para colocar la escalera de emergencia y
abrir las puertas del tren, las mujeres y los niños fueron las primeras en salir, la chica que
estaba frente a mi fue de las primeras en salir, nos dijeron que a pesar de que las vías no se
encotraban electrizadas, tuvieramos cuidado y pisaramos al centro de la vía, sobre la madera y el
carbón, y que alumbraramos nuestro camino con las lámparas de nuestros teléfonos si era
posible.Caminar en el túnel era fue algo raro, todo a nuestro alrededor estaba oscuro, lo único que
nos permitia seguir nuestro camino era el pequeño has de luz que nuestros teléfonos nos daban,
todos trataban de caminar lo más rápido posíble, algunos incluso caminaron sobre la parte exterior
de la vías para rebasar a los demás a pesar de las indicaciones de seguirdad, nadie quería seguir
bajo ese frío y oscuro túnel. Despúes de 10 minutos caminando en las vías pude observar el primer
rayo de lúz del exterior, conforme nos acercabamos al exterior mis ojos tenían algo de trabajo
ajustandose a la luz del exterior, despúes de todo casí habia estado una hora a oscuras en el vagon,
subímos unas escaleras para llegar al anden de la estación Pantitlan, el servicio había sido
suspendido, toda la estación había sido evacuada, solo nosotros faltabamos, subí las escaleras hacía
el puente que lleva a las calles de la colonia, toda la estación nunca había estado tan sola,
mientras caminaba me preguntaba qué tan mal había sido el temblor, hasta ahora no había visto nada
que me hiciera pensar que fue algo mayor pero todo era incertidumbre, llegue a la colonia y parecía
un sábado en la mañana, silencio en todos lados, poca gente, no había tráfico, tome mi transporte de
regreso a casa. Una vez que llegue a casa no había luz, no había teléfono móvil, encendí el radio y
comence a escuchar el reporte de la situación, definitivamente la ciudad no iba a ser la
misma por un buen rato.
\end{document}
